\subsection{Definiciones}

\begin{framed}
	\begin{description}
		\item [Número:] \cn
		\item [Nombre:] Identificadores.
		\item [Categoría:] Definiciones.
		\item [Descripción:]  El interprete debe facilitar mecanismos para que el usuario defina e identifique expresiones, datos, bloques de sentencias, y
		otras construcciones y elementos del lenguaje. Se precisa una manera unívoca de nombrar estos elementos. Un identificador válido debe estar
		formado por una secuencia de caracteres alfanuméricos de al menos un carácter, donde el primer carácter a de ser una letra.
	\end {description}
\end{framed}

\begin{framed}
	\begin{description}
		\item [Número:] \cn
		\item [Nombre:] Tabla de símbolos.
		\item [Categoría:] Definiciones.
		\item [Descripción:] El interprete debe ser capaz de gestionar tablas de símbolos. Los símbolos
		hacen referencias a valores, funciones y otras expresiones del lenguajes. Para acceder a estos símbolos
		se debe utilizar un identificador. Se hace necesario el acceso y uso de los símbolos según el contexto
		de ejecución, determinado por el ámbito y el tipo símbolo, para ello deben poder coexistir diferentes
		tablas de símbolos globales. Para evitar conflictos en el uso de identificadores algunos conceptos 
      deben disponer de su propia tabla de símbolos.
		
	\end {description}
\end{framed}

\subsubsection{Variables}
\begin{framed}
	\begin{description}
		\item [Número:] \cn
		\item [Nombre:] Variables.
		\item [Categoría:] Definiciones.
		\item [Descripción:] El interprete debe ser capaz gestionar una serie de símbolos
		denominados variables. Estos relacionan un identificador con un valor que puede variar durante el proceso de ejecución. El tipo
		de una variable dependerá del tipo del valor al que referencia (tipado dinámico), este podría ser de cualquiera de los tipos de
		datos soportados. La tabla de símbolos de variables debe adaptarse al contexto de ejecución.
	\end {description}
\end{framed}

\begin{framed}
	\begin{description}
		\item [Número:] \cn
		\item [Nombre:] Variables globales.
		\item [Categoría:] Definiciones.
		\item [Descripción:] Aunque la tabla de símbolos de variables es dependiente del contexto de ejecución se
		ha de facilitar algún mecanismo para que un dato esté disponible independientemente del contexto en el que
		se acceda.
	\end {description}
\end{framed}

\subsubsection{Funciones}

\paragraph{Defición de función}
\begin{framed}
	\begin{description}
		\item [Número:] \cn
		\item [Nombre:] Definición de función
		\item [Categoría:] Definiciones.
		\item [Descripción:] Se necesita de un mecanismo que permita definir y nominar bloques de sentencias. Estos bloques podrán recibir unos valores
		de entrada y producir una salida. Las sentencias en el bloque podrán operar sobre los parámetros de entrada, representados por
		unos símbolos variables que tomarán distintos valores en cada ejecución. Tras interpretarse el bloque 
		de sentencias se podrá tomar un valor considerado de salida. 
		
		La definición de una función representará en si misma un dato, por lo que podrán formar parte de operaciones 
		y otras expresiones. 
		Una función se define mediante un bloque de sentencias, una lista de identificadores que nominan a los parámetros de entrada y un 
		identificador que le da nombre a la propia función, aunque este último no debe ser necesario (funciones anónimas). 
		
	\end{description}
\end{framed}

\paragraph{Llamada a función}
\begin{framed}
	\begin{description}
		\item [Número:] \cn
		\item [Nombre:] Llamada a función
		\item [Categoría:] Definiciones.
		\item [Descripción:] Dada una función, se debe disponer de un mecanismo que permita la ejecución del bloque de 
		sentencias que la forma, mediante el uso de unos valores concretos como parámetros de entrada, y con la posibilidad de tomar 
		el valor de salida.
		
		Una llamada a función se deberá componer de un identificador relativo a su definición, y una lista de expresiones
		que determinarán los valores de los parámetros. La llamada deberá ser en si misma una expresión 
		que tomará como valor la salida de la función tras la ejecución.
		
		Los valores de los parámetros se corresponderán con los parámetros de la definición de la función de forma posicional.
	\end{description}
\end{framed}

\paragraph{Valor de retorno}
\begin{framed}
	\begin{description}
		\item [Número:] \cn
		\item [Nombre:] Valor de retorno
		\item [Categoría:] Definiciones.
		\item [Descripción:] Se necesita de un mecanismo en forma de sentencia que, dada una función, determine
		el valor del salida que se tomará en la llamada a la misma. La sentencia return se compondrá de una 
		expresión correspondiente al valor salida. Al ser interpretada esta sentencia la ejecución de
		la función deberá finalizar y esta tomará el valor de la expresión dada.
	\end{description}
\end{framed}

\paragraph{Valores de parámetros por defecto}
\begin{framed}
	\begin{description}
		\item [Número:] \cn
		\item [Nombre:] Valores de parámetros por defecto.
		\item [Categoría:] Definiciones.
		\item [Descripción:] Dada la definición de una función, debe existir un mecanismo para que los parámetros de esta puedan tener 
		valores por defecto. Estos valores serán asignado a los parámetros cuando en una llamada a función no sean determinados. 
		
		Como la correspondencia entre parámetros en una llamada a función se hace de forma posicional, los valores por defecto deberán ser especificados 
		desde el final de la lista de parámetros hasta el inicio.
	\end{description}
\end{framed}

\paragraph{Parámetros por valor}
\begin{framed}
	\begin{description}
		\item [Número:] \cn
		\item [Nombre:] Parámetros por valor.
		\item [Categoría:] Definiciones.
		\item [Descripción:] Cuando una función es ejecutada todos los símbolos variables que se definan y utilicen deben tratarse de forma 
		local al bloque de sentencias de la función. De esta forma los símbolos variables definidos fuera de la función no serán accesibles
		desde el cuerpo de la misma y viceversa. Cuando se realice una llamada a función los valores de los parámetros deben ser copiados
		a los símbolos variables correspondientes.
	\end{description}
\end{framed}

\paragraph{Parámetros por referencia}
\begin{framed}
	\begin{description}
		\item [Número:] \cn
		\item [Nombre:] Parámetros por referencia.
		\item [Categoría:] Definiciones.
		\item [Descripción:] Se necesita de un mecanismo que permita que los parámetros de una función referencien valores definidos 
		fuera del cuerpo de la misma. De esta forma se podrá acceder y/o modificar datos externos a la función. 
		
		Cuando en una llamada a función se especifiquen expresiones que sean símbolos variables como algunos de sus parámetros, si estos se definieron en la
		función como parámetros por referencia, el valor del símbolo en la llamada será referenciado por el símbolo correspondiente de la función.
	\end{description}
\end{framed}

\paragraph{Función lambda}
\begin{framed}
	\begin{description}
		\item [Número:] \cn
		\item [Nombre:] Función lambda.
		\item [Categoría:] Definiciones.
		\item [Descripción:] Se debe dar la posibilidad de crear funciones anónimas. Estas funciones carecerán de nombre y 
		normalmente se utilizarán en la asignación de variables, como parámetros de otras funciones o como valor de retorno. 
		Las funciones lambda deberán ser en si misma una expresión que toma como valor el dato correspondiente a la función. 
	\end{description}
\end{framed}

\paragraph{Función lambda simple}
\begin{framed}
	\begin{description}
		\item [Número:] \cn
		\item [Nombre:] Función lambda simple.
		\item [Categoría:] Definiciones.
		\item [Descripción:] Se debe de facilitar un mecanismo para crear funciones lambdas simples, que solo consten de una lista de parámetros
		y de una única expresión que será devuelta y que constituirá el cuerpo de la función. 
	\end{description}
\end{framed}

\paragraph{Referencia a función}
\begin{framed}
	\begin{description}
		\item [Número:] \cn
		\item [Nombre:] Referencia a función.
		\item [Categoría:] Definiciones.
		\item [Descripción:] Se debe facilitar un mecanismo para referenciar funciones ya creadas, de forma que puedan ser asignadas a variables, 
		pasadas como parámetros o devueltas como valor de retorno. 
	\end{description}
\end{framed}

\paragraph{Funciones de orden superior}
\begin{framed}
	\begin{description}
		\item [Número:] \cn
		\item [Nombre:]  Funciones de orden superior.
		\item [Categoría:] Definiciones.
		\item [Descripción:] Se debe poder definir funciones de orden superior, estas pueden recibir otras funciones como parámetros o tomar como valor de 
      retorno otra función.  
	\end{description}
\end{framed}

\paragraph{Funciones clausura}
\begin{framed}
	\begin{description}
		\item [Número:] \cn
		\item [Nombre:]  Funciones clausura.
		\item [Categoría:] Definiciones.
		\item [Descripción:] Se debe poder definir funciones dentro del contexto de otras. La función de clausura puede tener variables que dependan 
      del entorno en el que se ha definido la función. De esta forma cuando la función es llamada podrá acceder al valor de la variable en el contexto
      en el que se definió.
	\end{description}
\end{framed}

\paragraph{Aplicacion parcial}
\begin{framed}
	\begin{description}
		\item [Número:] \cn
		\item [Nombre:] Aplicación parcial.
		\item [Categoría:] Definiciones.
		\item [Descripción:] Se debe facilitar un mecanismo que permita, a partir de una función, obtener otra
      equivalente donde se ha dado valor a un subconjunto de los parámetros.
    \end{description}
\end{framed}

\paragraph{Decoradores}
\begin{framed}
	\begin{description}
		\item [Número:] \cn
		\item [Nombre:] Decoradores.
		\item [Categoría:] Definiciones.
		\item [Descripción:] Se debe facilitar un mecanismo para definir decoradores. Un decorador será un tipo especial de función.
      Al igual que una función se define mediante un identificador que lo nomina, una lista de parámetros y un bloque de sentencias. 
      
      A diferencia de las funciones ordinarias, la llamada a un decorador deberá tener como parámetro una función que será decorada, como
      resultado se deberá obtener una función que tendrá las siguientes características:
      
      \begin{itemize}
         \item La lista de parámetros que admite será la misma que la lista con la que se definió el decorador
         \item El bloque de sentencias será el del decorador pero haciendo uso de la función que ha sido decorada
      \end{itemize}
      
      Se debe facilitar un mecanismo para referenciar la función que se va a decorar dentro del decorador. Para ello
      se utilizará la función de contexto.
	\end{description}
\end{framed}

\paragraph{Función de contexto}
\begin{framed}
	\begin{description}
		\item [Número:] \cn
		\item [Nombre:] Función de contexto.
		\item [Categoría:] Definiciones.
		\item [Descripción:] Se debe facilitar un mecanismo para acceder a la función de contexto. Esta será una función 
      cuyo valor dependerá del contexto en el que se ejecute:
      \begin{itemize}
         \item En el primer nivel de ejecución la función de contexto no estará definida. 
         \item En el cuerpo de una función será la propia función.  
         \item En el cuerpo de un decorador será la función que se decorará.
      \end{itemize}
	\end{description}
\end{framed}



\subsubsection{Clases de objetos}
\paragraph{Clase de objeto}
\begin{framed}
	\begin{description}
		\item [Número:] \cn
		\item [Nombre:] Clase de objeto.
		\item [Categoría:] Definiciones.
		\item [Descripción:] El lenguaje debe contemplar el paradigma de la programación orientada a objetos. Una clase
      se ha de ver como una definición estática e inmutable que será utilizada para la creación de objetos.
      
      Las clases definirán tipos de objetos que tendrán métodos y atributos comunes. Una clase se construye
		mediante un identificador que le da nombre y un bloque de sentencias que contendrá una serie de funciones (métodos)
		y símbolos variables (atributos).  
      
      Las características de la programación orientada a objetos que se deberán contemplar son:
		\begin{description}
			\item [Abstracción:] Un objeto por si mismo representará una entidad abstracta que podrá tener cierta funcionalidad
			asociada, disponer de atributos que establezcan su estado interno o comunicarse con otros objetos. 
			\item [Encapsulamiento:] Un objeto podrá contener todos los elementos correspondiente a su definición, estado y funcionalidad.
         \item [Principio de ocultación:] Un objeto podrá tener atributos y/o métodos privados, de forma que sólo sean
         accesibles desde el propio objeto. 
			\item [Polimorfismo:] Se debe permitir a objetos de distinto tipo se le pueda enviar mensajes sintácticamente iguales, de forma
			que se pueda  llamar un método de objeto sin tener que conocer su tipo.
			\item [Herencia:] Se debe contemplar la herencia simple entre clases de forma que  una clase se pueda definir mediante otra.
		\end{description}
	\end {description}
\end{framed}

\paragraph{Objeto}
\begin{framed}
	\begin{description}
		\item [Número:] \cn
		\item [Nombre:] Objeto.
		\item [Categoría:] Definiciones.
		\item [Descripción:] El sistema debe ser capaz de interpretar y operar sobre objetos. Los objetos serán vistos
		como estructuras de datos funcionales, formado tanto por datos de diferentes tipos (atributos), como por funciones (métodos). Un objeto
		puede o no pertenecer a una clase de objetos. 
      
      Se podrá crear objetos a partir de una clase ya definida, mediante la clonación de otros objeto, o directamente
		mediante sentencias y expresiones. Las clases de objetos pueden definir un método constructor que será utilizado cuando se cree un
		objeto a partir de la misma. 
      
      Dentro del bloque de sentencias que conforma un método es posible hacer referencia a los demás valores del objeto mediante 
      la expresión especial ``this''.
      
      Las clases de objetos podrán definir un método constructor que será llamado cuando el objeto sea instanciado. 
      
      Un objeto podrá disponer de un método que será llamado cuando se precise su conversión a un tipo de dato cadena de caracteres.
	\end {description}
\end{framed}

\paragraph{Elementos privados}
\begin{framed}
	\begin{description}
		\item [Número:] \cn
		\item [Nombre:] Elementos privados.
		\item [Categoría:] Clases de Objetos.
		\item [Descripción:] Se debe facilitar un mecanismo para definir atributos y métodos de una clase de objetos como privados. 
      Estos elementos solo serán accesibles desde métodos del propio objeto. Se deberá contemplar el acceso a estos elementos
      sobre objetos del mismo tipo dentro de métodos de la clase.
	\end{description}
\end{framed}

\paragraph{Elementos estáticos}
\begin{framed}
	\begin{description}
		\item [Número:] \cn
		\item [Nombre:] Elementos estáticos.
		\item [Categoría:] Clases de Objetos.
		\item [Descripción:] Se debe facilitar un mecanismo para definir atributos y métodos de una clase de objetos pertenecientes
      a la propia clase. Estos elementos no serán trasladados a los objetos instanciados.
	\end{description}
\end{framed}

\paragraph{Herencia de clases}
\begin{framed}
	\begin{description}
		\item [Número:] \cn
		\item [Nombre:] Herencia de clases.
		\item [Categoría:] Clases de Objetos.
		\item [Descripción:] Se debe de disponer de un mecanismo que permita establecer una relación de herencia entre unas clases dadas. 
		Así será posible la definición de nuevas clases partiendo de otras. La clase derivará de otra extendiendo su funcionalidad y definición. 
		
		En la definición de una clase se debe de disponer de un mecanismo que permita especificar la clase que se extenderá. La nueva clase tendrá todos 
		los atributos y métodos de la extendida y añadirá los suyos propios, pudiendo sobrescribirse los ya existentes.
	\end{description}
\end{framed}

\paragraph{Instanciación de clases}
\begin{framed}
	\begin{description}
		\item [Número:] \cn
		\item [Nombre:] Instanciación de clases.
		\item [Categoría:] Clases de Objetos.
		\item [Descripción:] Dada una clase, se debe de disponer de un mecanismo que permita crear objetos a partir de la misma.   
		Para construir un objeto a partir de la instanciación de una clase se deben llevar las funciones y variables definidas en el
		cuerpo de la clase a métodos y atributos del objeto. 
		
		Una clase pude definir un método constructor que deberá ser llamado sobre el objeto recién creado cuando la clase es instanciada.
		
		La instanciación se deberá realizar mediante un operador que, a partir de un identificador correspondiente a la clase y una lista de expresiones 
		correspondientes a los parámetros del método constructor, tome como valor el objeto recién creado.
	\end{description}
\end{framed}


\paragraph{Acceso al objeto en ejecución}
\begin{framed}
	\begin{description}
		\item [Número:] \cn
		\item [Nombre:] Acceso al objeto en ejecución.
		\item [Categoría:] Clases de Objetos.
		\item [Descripción:] Se debe de disponer de un mecanismo que permita acceder a los atributos y métodos de un objeto desde la ejecución 
		de un método del mismo. Este mecanismo, correspondiente a una expresión, deberá tomar como valor el objeto en ejecución. 
	\end{description}
\end{framed}

\paragraph{Acceso al objeto en ejecución como clase padre}
\begin{framed}
	\begin{description}
		\item [Número:] \cn
		\item [Nombre:] Acceso al objeto en ejecución como clase padre.
		\item [Categoría:] Clases de Objetos.
		\item [Descripción:] Se debe de disponer de un mecanismo que permita acceder a los atibutos y métodos de la clase padre de un objeto desde 
      la ejecución de un método del mismo. Este mecanismo, correspondiente a una expresión, deberá tomar como valor el objeto en ejecución, pero tomando
      como métodos y atributos los de la clase padre de la cual deriva. 
	\end{description}
\end{framed}

\paragraph{Enlace estático en tiempo de ejecución}
\begin{framed}
	\begin{description}
		\item [Número:] \cn
		\item [Nombre:] Enlace estático en tiempo de ejecución.
		\item [Categoría:] Clases de Objetos.
		\item [Descripción:] Se debe de disponer de un mecanismo que permita acceder a los atibutos y métodos estáticos de una clase hija desde un método 
      estático de la clase padre. 
	\end{description}
\end{framed}




\subsubsection{Listas}

\begin{framed}
	\begin{description}
		\item [Nombre:] Listado.
		\item [Categoría:] Definiciones.
		\item [Número:] \cn
		\item [Descripción:] Se ha de facilitar un mecanismo que permita agrupar expresiones para darles un significado operacional
		común. Este deberá consistir en una serie de expresiones separadas por comas.
	\end {description}
\end{framed}

\subsubsection{Pares clave/valor}
\begin{framed}
	\begin{description}
		\item [Nombre:] Pares clave/valor.
		\item [Categoría:] Definiciones.
		\item [Número:] \cn
		\item [Descripción:] Se ha de facilitar un mecanismo que permita relacionar un par de expresiones para darles un significado estructural.
		Este deberá consistir en el par de expresiones separadas por el carácter ``:''.
	\end {description}
\end{framed}

\subsubsection{Etiqueta}
\begin{framed}
	\begin{description}
		\item [Número:] \cn
		\item [Nombre:] Etiqueta.
		\item [Categoría:] Definiciones.
		\item [Descripción:] El sistema debe ser capaz de interpretar y operar sobre datos de tipo etiqueta. Una etiqueta es una referencia
		a una sentencia concreta dentro del contenido fuente. Las etiquetas dependen quedarán definidas dentro de un contexto determinado por
		el bloque de sentencias en el que se encuentren.
	\end {description}
\end{framed}

\subsubsection{Generadores}
\begin{framed}
	\begin{description}
		\item [Número:] \cn
		\item [Nombre:] Generador.
		\item [Categoría:] Definiciones.
		\item [Descripción:] Se necesita de un mecanismo que sea una expresión por si mismo y que permita generar arrays desde una sentencia iterativa. 
		Este mecanismo se formará mediante una expresión seguida da una sentencia for. La expresión será ejecutada tras iteración del bucle y será 
		asignada como último elemento de un array. Al final de la ejecución la expresión tomará el valor del array generado.
	\end{description}
\end{framed}

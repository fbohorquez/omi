\subsection{Ficheros}
\subsubsection{Fichero}
\begin{framed}
	\begin{description}
		\item [Número:] \cn
		\item [Nombre:] Fichero.
		\item [Categoría:] Operadores sobre ficheros.
		\item [Descripción:] El intérprete debe ser capaz de manipular ficheros, para ello se precisa de un tipo de dato
		que simbolice un puntero a un fichero del sistema de ficheros. Este tipo de dato no debe ser convertido a ningún otro
		tipo de dato ni viceversa. Además solo será tratado por algunos operadores dedicados. 
		
		No se tendrán en cuenta los  ficheros binarios.
	\end{description}
\end{framed}

\subsubsection{Abrir ficheros}
\begin{framed}
	\begin{description}
		\item [Número:] \cn
		\item [Nombre:] Abrir ficheros.
		\item [Categoría:] Operadores sobre ficheros.
		\item [Descripción:] Es necesario un operador que permita abrir ficheros para su manipulación. Este tendrá como operandos
		una cadena de caracteres que simbolice la ruta al fichero y otra que determine el modo en el que será abierto. El operador 
		tomára como valor un dato de tipo puntero a fichero. Los posibles modos serán:
		
		\begin{description}
			\item [r:] Lectura.
			\item [r+:] Lectura y/o escritura.
			\item [w:] Escritura truncando el contenido del fichero.
			\item [w+:] Lectura y/o escritura truncando el contenido del fichero.
			\item [a:] Escritura posicionando el puntero al final el fichero.
			\item [a+:] Lectura y/o escritura posicionando el puntero al final del fichero.
		\end{description}
		
		Todos los modos a excepción de sólo lectura deberán crear el fichero si este no existe.  
	\end {description}
\end{framed}

\subsubsection{Cerrar ficheros}
\begin{framed}
	\begin{description}
		\item [Número:] \cn
		\item [Nombre:] Cerrar ficheros.
		\item [Categoría:] Operadores sobre ficheros.
		\item [Descripción:] Es necesario un operador que permita cerrar ficheros abiertos a partir de un puntero al mismo.
		Se deberá finalizar cualquier flujo de datos abierto y el fichero quedará cerrado. Como valor se deberá tomar 
		un dato de tipo lógico que determine si la operación se ha realizado correctamente.
	\end {description}
\end{framed}

\subsubsection{Escribir en fichero}
\begin{framed}
	\begin{description}
		\item [Número:] \cn
		\item [Nombre:] Escribir en fichero.
		\item [Categoría:] Operadores sobre ficheros.
		\item [Descripción:] Se hace necesario un operador que, dado un dato de tipo puntero a fichero, pueda escribir datos
		en la posición referenciada por el mismo. Así este operador trabaja sobre dos operandos, un puntero a fichero y una 
		cadena de caracteres que simbolizará el contenido a escribir. Como valor el operador toma el número de bytes que 
		fueron escritos.
	\end{description}
\end{framed}

\subsubsection{Leer de fichero}
\begin{framed}
	\begin{description}
		\item [Número:] \cn
		\item [Nombre:] Leer de fichero.
		\item [Categoría:] Operadores sobre ficheros.
		\item [Descripción:] Se hace necesario un operador que, dado un dato de tipo puntero a fichero, lea desde la
		posición referencia por el mismo hasta un carácter de nueva línea, o bien un número de carácteres dado. Así el operador deberá tomar como valor 
		una cadena de caracteres que represente el contenido leído.
	\end{description}
\end{framed}

\subsubsection{Cambiar posición de puntero a fichero}
\begin{framed}
	\begin{description}
		\item [Número:] \cn
		\item [Nombre:] Cambiar posición de puntero a fichero.
		\item [Categoría:] Operadores sobre ficheros.
		\item [Descripción:] Una operación básica sobre punteros a ficheros es desplazar este dentro del contenido del mismo. 
		Para ello se precisa de un operador que, dado un puntero a fichero, cambie la posición de este dentro del propio fichero.
		Así la nueva posición deberá ser una expresión numérica que represente un offset relativo al principio del fichero, el final 
		o la posición actual del puntero. La expresión correspondiente al operador deberá tomar un valor booleano que determine si el cambio
		de posición se ha realizado correctamente.
	\end{description}
\end{framed}

\subsubsection{Obtener la posición actual de puntero a fichero}
\begin{framed}
	\begin{description}
		\item [Número:] \cn
		\item [Nombre:] Obtener la posición actual de puntero a fichero.
		\item [Categoría:] Operadores sobre ficheros.
		\item [Descripción:] Se necesita de un operador que dado un puntero a fichero tome el valor aritmético que represente la posición de este 
		dentro del mismo.
	\end{description}
\end{framed}

\subsubsection{Obtener contenido de un fichero}
\begin{framed}
	\begin{description}
		\item [Número:] \cn
		\item [Nombre:] Obtener contenido de un fichero.
		\item [Categoría:] Operadores sobre ficheros.
		\item [Descripción:] Se precisa de un operador que simplifique la tarea de obtener el contenido completo de un fichero, sin que sea necesario disponer de un 
		puntero al mismo. Para ello se deberá facilitar una cadena de caracteres que simbolice la ruta completa del fichero. El operador tomará como valor
		una cadena de caracteres que contenga todo el contenido del fichero. En el caso de que el fichero no exista se deberá tomar como valor la cadena vacía.
	\end{description}
\end{framed}

\subsubsection{Cadena como contenido de un fichero.}
\begin{framed}
	\begin{description}
		\item [Número:] \cn
		\item [Nombre:] Cadena como contenido de un fichero.
		\item [Categoría:] Operadores sobre ficheros.
		\item [Descripción:] Se precisa de un operador que simplifique la tarea de escribir una cadena de caracteres en un fichero, sin que sea necesario 
		disponer de un puntero al mismo. Si el fichero existe su contenido deberá ser truncado, si no existe será creado. Este operador tendrá como operandos dos cadenas 
		de caracteres que se correspondan con la ruta del fichero y la cadena a escribir. Como valor se tomára la cadena escrita.
	\end{description}
\end{framed}

\subsubsection{Añadir cadena al contenido de un fichero}
\begin{framed}
	\begin{description}
		\item [Número:] \cn
		\item [Nombre:] Añadir cadena al contenido de un fichero.
		\item [Categoría:] Operadores sobre ficheros.
		\item [Descripción:] Se precisa de un operador que simplifique la tarea de añadir una cadena de caracteres al final de un fichero, sin que sea necesario 
		disponer de un puntero al mismo. Si el fichero no existe será creado. Este operador tendrá como operandos dos cadenas 
		de caracteres que se correspondan con la ruta del fichero y la cadena a escribir. Como valor se tomará la cadena escrita.
	\end{description}
\end{framed}

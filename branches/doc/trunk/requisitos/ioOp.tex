\subsection{Operadores de entrada/salida}

\subsubsection{Escribir en la salida estándar}
\begin{framed}
	\begin{description}
		\item [Número:] \cn
		\item [Nombre:] Escribir en la salida estándar.
		\item [Categoría:] Operadores de entrada/salida.
		\item [Descripción:] Se debe disponer de una sentencia que permita llevar acabo la salida de datos. Estas supondrán un mecanismo para
		que el contenido fuente pueda mostrar en la salida estándar los datos sobre los que opera. Estos datos
		debes tener una representación gráfica y ser imprimibles.  
	\end {description}
\end{framed}

\subsection{Leer de la entrada estándar}
\begin{framed}
	\begin{description}
		\item [Número:] \cn
		\item [Nombre:] Leer de la entrada estándar.
		\item [Categoría:] Operadores de entrada/salida.
		\item [Descripción:] El interprete debe implementar algún recurso que permita que el contenido fuente del usuario
		obtenga datos de la entrada estándar para operar. Este mecanismo deberá dar la posibilidad de mostrar un prompt.
		Además debe dar la opción de que la entrada finalice al introducir un salto de línea o al
		generarse una señal EOF (end-of-file).
	\end {description}
\end{framed}

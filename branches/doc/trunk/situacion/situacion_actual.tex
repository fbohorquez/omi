\section {Visión general}
En esta sección se lleva a cabo un analisis de la situación actual en el estudio de la teoría de autómatas y los lenguajes formales. 
Se describirán las necesidades presentes en este campo, y que serán objetivo de las soluciones planteadas. A partir de estos objetivos se llevará 
acabo una descripción de los requisitos que debe cumplir la solución tomada. 

\section{Situación actual}

El estudio de los lenguajes formales es anterior a la concepción de las computadoras. Las matemáticas, la lógica y otras ciencias venían haciendo 
uso de los conceptos de los lengujes formales para la solución de problemas. 

Una computadora desde un punto de vista teórico es un autómata o máquina de estados, que es capaz de ejecutar una serie de instrucciones descritas
en lenguaje máquina. Por tanto fueron los avances en la rama de la teoría de autómata y los lenguajes formales, entre otros campos, los que permitieron la concepción de 
las primeras computadoras. Actualmente la mayoría de las ingienerías de la información se estudian los conceptos teóricos detras de las computadoras 
y los lenguajes de programación.

Lo más común es que un estudiante de informática comience sus estudios en estos campos con los autómatas: los tipos que existen, cómo se definen 
y para qué se utilizán. Llegando a comprender conceptos como el de estado, alfabeto, etc. Incluso estudiando definiciones de algunos tipos de lenguajes 
formales como los lenguajes regulares.  

Posteriormente el alumno podría utilizar los conceptos aprendidos para estudiar distintos modelos de computación, los cuales son definidos formalmente y desde un punto de vista teórico.
La máquina de turing y el cálculo lambda son piezas escenciales en este punto del aprendizaje. 

Una vez que se es poseedor del conocimiento base en la teoría de autómatas y los lenguajes formales, el alumno puede aplicar estos conociminetos para estudiar las estructuras, mecanismos 
y demás conceptos detrás de los lengujes de programación. En este punto se estudian los interpretes y compiladores, y los 
conceptos básicos que hay detrás estos. Conceptos como el léxico, sintaxis y la semantica de los lenguajes de programación, las tablas de símbolos, o las distintos tipos de gramáticas. 

Hasta aquí se habrá obtenido los conocimientos teóricos necesarios y el alumno podría comenzar a dar soluciones prácticas a problemas, aplicando así los conociminetos obtenidos. Así es
común que se comience con el desarrollo de analziadores léxicos y sintácticos sencillos y concretos, para luego aplicarles una semántica. Ejercicios como el desarrollo de 
una calculadora suelen ser habituales. Además se estudian algunas de las herramientas que asisten al proceso del desarrollo de este tipo de tecnologías.

Después del proceso descrito al alumno se le ha brindado la oportunidad de profundizar en un campo con multitud de ramas, técnicas, metodologías y conceptos, que son fruto de años de 
estudio de expertos y apasionados. Podría profundizar en el proceso de compilación o traducción, en las distintas gramáticas, en técnicas de optimización o diseñar sus propias herramientas 
de traducción o interpretación de lenguajes formales.

Por otro lado, en la industria de la técnología de la información se hace uso de lenguajes muy completos, con gran diversidad de mecanismos y bien consolidados, que son efecto de la evolución 
y las necesidades en el sector. En la mayoría de cursos académicos se estudian estas herramientas desde un punto de vista práctico y de uso. 

La teoría de autómatas y lenguajes formales presenta la base para el estudio de los compiladores e interpretes que son parte fundamental de la industrial 
actual. A pesar de ello no existen herramientas divulgativas, colaborativas e interactivas que, a partir de los conocimientos básicos, ayude a comprender cómo se desarrollan los distintos 
mecanismos y herramientas presentes en la tecnología actual.

Los lenguajes de programación han evolucionado mucho desde que comenzaron ha alejarse del lenguaje máquina. El alumnado actual trabaja sobre los conceptos que le ayudan a entender esta evolución, 
pero no dispone de herramientas o medios para ver cómo estos conceptos son trasladados a un producto real y presente en el día a día de un programador actual. 

\section{Necesidades}
Dada la situación actual se precisa de una herramienta que ayude a comprender cómo se implementa y construye un intérprete para un lenguaje de programación. Es condición necesaria que 
este proceso quede correctamente documentado. La herramienta elaborada deberá ser accesible por cualquier interesado en el tema, que desee profundizar en la práctica del desarrollo
de interpretes y lenguajes de programación. 

Se partirá de los conceptos básicos de la teoría de autómatas y los lenguajes formales, así como de la teoría de compiladores e intérpretes. Se asume pues que el usuario dispone
de este conocimiento.

\section{Objetivos}
Se llegará a construir un lenguaje de programación completo con características presente en la tecnología actual. Este proceso quedará correctamente documentado y se pondrá a disposición pública.
Las características que serán contempladas son:

\begin{itemize}
\item Distintos tipos de datos simples y compuestos.
\item Expresiones lógicas y aritméticas.
\item Expresiones y funciones sobre cadenas.
\item Expresiones y funciones sobre vectores de datos.
\item Expresiones y funciones sobre expresiones regulares.
\item Operadores de conversión de tipos.
\item Mecanismos de entrada/salida.
\item Creación de procesos.
\item Manipulación de ficheros.
\item Funciones de fecha y tiempo.
\item Definición y uso de varibles de tipado dinámico.
\item Ámbito de variables.
\item Sentencias de control de flujo condicionales e iterativas.
\item Definicnión y uso de funciones y procedimientos.
\item Mecanismos y técnicas de la programación funcional.
\item Definición y uso de clases de objetos.
\item Mecanismos y técnicas de la programación orientada a objeto.
\item Integración de módulos que extienden el lenguaje.
\end{itemize}

El interprete desarrollado podrá ser usado de una forma interactiva, permitiendo así la ejecución de
instrucciones bajo demanda. 

Toda la documentación generada a partir del proceso de desarrollo deberá ser estructurada y cumplimentada para 
formar una parte de una aplicación web que la haga accesible. La aplicación web además permitirá el uso online del 
intérprete.

El proyecto presentará una licencia de uso libre para que pueda ser usado abiertamente por la comunidad,  a la vez que 
se nutre de las contribuciones de la misma. 




  
 

 
 

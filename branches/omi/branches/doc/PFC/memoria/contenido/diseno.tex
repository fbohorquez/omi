% ------------------------------------------------------------------------------
% Este fichero es parte de la plantilla LaTeX para la realización de Proyectos
% Final de Grado, protegido bajo los términos de la licencia GFDL.
% Para más información, la licencia completa viene incluida en el
% fichero fdl-1.3.tex

% Copyright (C) 2012 SPI-FM. Universidad de Cádiz
% ------------------------------------------------------------------------------

En esta sección se recoge la arquitectura general del sistema de información, la parametrización del software base (opcional), el diseño físico de datos, el diseño detallado de componentes software y el diseño detallado de la interfaz de usuario.

\section{Arquitectura del Sistema}
En esta sección se define la arquitectura general del sistema de información, especificando la infraestructura tecnológica necesaria para dar soporte al software y la estructura de los componentes que lo forman.

\subsection{Arquitectura Física}
En este apartado, describimos los principales elementos hardware que forman la arquitectura física de nuestro sistema, recogiendo por un lado los componentes del entorno de producción y los componentes de cliente.\\

Se debe incluir un modelo de despliegue en el cual se describe cómo los elementos software son desplegados en los elementos hardware. También se incluyen las especificaciones y los requisitos del hardware (servidores, etc.), así como de los elementos software (sistemas operativos, servicios, aplicaciones, etc.) necesarios.

\subsection{Arquitectura Lógica}
La arquitectura de diseño especifica la forma en que los artefactos software interactúan entre sí para lograr el comportamiento deseado en el sistema. En esta sección se muestra la comunicación entre el software base seleccionado, los componentes reutilizados y los componentes desarrollados para cumplir los requisitos de la aplicación. También, se recogen los servicios de sistemas externos con los que interactúa nuestro sistema.
Se debe incluir un diagrama de componentes que muestre en un alto nivel de abstracción los artefactos que conforman el sistema.\\

Existen diferentes patrones o estilos arquitectónicos. En los sistemas web de información es común la utilización del patrón Layers (Capas), con el cual estructuramos el sistema en un número apropiado de capas, de forma que todos los componentes de una misma capa trabajan en el mismo nivel de abstracción y los servicios proporcionados por la capa superior utilizan internamente los servicios proporcionados por la capa inmediatamente inferior. Habitualmente se tienen las siguientes capas:

\paragraph*{Capa de presentación (frontend)}
Este grupo de artefactos software conforman la capa de presentación del sistema, incluyendo tanto los componentes de la vista como los elementos de control de la misma.

\paragraph*{Capa de negocio}
Este grupo de artefactos software conforman la capa de negocio del sistema, incluyendo los elementos del modelo de dominio y los servicios (operaciones del sistema).

\paragraph*{Capa de persistencia}
Este grupo de artefactos software conforman la capa de integración del sistema, incluyendo las clases de abstracción para el acceso a datos (BD o sistema de ficheros) o a sistemas heredados.\\

Es común que a la capa de negocio y de datos de los sistemas web, se denomine conjuntamente como backend o modelo de la aplicación.

Opcionalmente, podemos disponer de un conjunto de artefactos software que pueden ser usados por elementos de cualquiera de las capas del sistema y que fundamentalmente proporcionan servicios relacionados con requisitos no funcionales (calidad).

\section{Parametrización del software base}
En esta sección, se detallan las modificaciones a realizar sobre el software base, que son requeridas para la correcta construcción del sistema. En esta sección incluiremos las  actuaciones necesarias sobre la interfaz de administración del sistema, sobre el código fuente o sobre el modelo de datos.

\section{Diseño Físico de Datos}
En esta sección se define la estructura física de datos que utilizará el sistema, a partir del modelo de conceptual de clases, de manera que teniendo presente los requisitos establecidos para el sistema de información y las particularidades del entorno tecnológico, se consiga un acceso eficiente de los datos.
La estructura física se compone de tablas, índices, procedimientos almacenados, secuencias y otros elementos dependientes del SGBD a utilizar.

\section{Diseño detallado de Componentes}
Para cada uno de los módulos funcionales del sistema debemos realizar un diagrama de secuencia, para definir la interacción existente entre las clases de objetos que permitan responder a eventos externos.

\section{Diseño detallado de la Interfaz de Usuario} 
En esta sección se detallarán las interfaces entre el sistema y el usuario, incluyendo un prototipo de alta fidelidad con el diseño de la IU. Se definirá el comportamiento de las diferentes pantallas, indicando qué ocurre en los distintos componentes visuales de la interfaz cuando aparecen y qué acciones se disparan cuando el usuario trabaja con ellas.


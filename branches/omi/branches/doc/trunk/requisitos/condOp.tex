\subsection{Operadores condicionales}
\subsubsection{Operador ternario}
\begin{framed}
	\begin{description}
		\item [Número:] \cn
		\item [Nombre:] Operador ternario.
		\item [Categoría:] Operadores condicionales.
		\item [Descripción:] Se ha de contemplar el operador ``ternario''. Este constará de tres expresiones que se denominarán
		``condición'', ``caso verdadero'' y ``caso negativo''. Primero se deberá evaluar la expresión ``condición'' como un
		dato lógico, si esta es positiva el valor de la operación será el de la expresión ``caso verdadero'', si es negativa
		se tomará el de ``caso falso''.
		
		Se debe facilitar formas del operador ternario simplificadas en las que falten algunos
		de los operandos. Así se contemplará el ternario que carecerá de ``caso verdadero'' tomándose en su lugar
		el valor de la expresión ``condición'' (sin alterar el tipo de dato). También el ternario que carece de ``caso falso''
		tomándose en su lugar el valor de la cadena vacía.
	\end {description}
\end{framed}

\subsubsection{Operador fusión de nulos}
\begin{framed}
	\begin{description}
		\item [Número:] \cn
		\item [Nombre:] Operador fusión de nulos.
		\item [Categoría:] Operadores condicionales.
		\item [Descripción:] Se ha de contemplar el operador ``fusión de nulos''. Este consistirá en una lista de expresiones que serán
		evaluadas de forma secuencial y lógica, tomándose como valor el de la primera expresión
		cuya evaluación sea positiva, o el valor nulo si no existe ninguna.
	\end {description}
\end{framed}

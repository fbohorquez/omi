\section{Requisitos no funcionales}
\subsection{Rendimiento}
Es condición necesaria que el sistema software desarrollado presente un rendimiento óptimo. Para ello se medirá el rendimiento 
desde dos aspectos: tiempo y espacio
\subsubsection{Tiempo}
A pesar de que el objetivo del interprete no es servir para la producción de software, al ser un sistema que será utilizado para el 
desarrollo de otras aplicaciones, el rendimiento en cuanto al tiempo debe ser aceptable en comparación con las herramientas existentes en 
el mercado. 

\subsubsection{Espacio}
El sistema debe ser óptimo en cuanto el espacio en memoria que ocupan sus estructras. Debe hacer un uso correcto de la memoria. 
Si las estructuras que conforman el intérprete ocupan demasiado espacio, los datos de los que hagan uso el usuario en sus programas también lo
harán.

\subsection{Usabilidad}
\subsubsection{Reglas léxicas y sintácticas'}
El sistema deberá interpretar un lenguaje con reglas léxicas y sintácticas claras, que sean fáciles de comprender y asimilar. La estructura del 
lenguaje en cuanto a estos dos aspectos debe ser similar a los lenguajes de programación actuales. 
\subsubsection{Interfaz}
La interfaz de entrada salida del interprete debe ser clara y presentar las opciones de una forma adecuada. Además la interfaz web debe 
ser amigable a la forma de operara del usuario que la utilice.

\subsection{Accesibilidad}
El sistema desarrollado debe ser accesible desde cualquier computadora con acceso a internet. Para ello se utilizará un navegador web. Por otro lado 
el interprete podrá ser instalado de forma local para un uso individual.

\subsection{Estabilidad}
Se requiere que el sistema desarrollado presente un umbral de fallos bajo. Debe adaptarse a los casos exepcionales, y en caso de error informar adecuadamente de
los motivos y causas del mismo.

\subsection{Mantenibilidad}
El sistema desarrollado debe ser mantenible en el tiempo. Para ello debe estar correctamente documentado, modularizado y estructurado. 

\subsection{Concurrencia}
El sistema deberá ser accesible por varios usuarios en el mismo marco de tiempo. Es por ello que deberá ser capaz de funcionar en un entorno de 
concurrencia.





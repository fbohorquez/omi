OMI representa una plataforma constituida por una serie de herramientas. Estas ayudan al aprendizaje en la aplicación práctica de la teoría
de autómatas y lenguajes formales para el desarrollo de un lenguaje de programación moderno. Sus características son las siguientes:

\begin{itemize}
\item Interprete OMI
   \begin {itemize}
   \item Abierto
   \item Modular
   \item Interactivo
   \item Configurable
   \item Paso de argumentos
   \end{itemize}
\item Lenguaje OMI
   \begin {itemize}
   \item Propósito general
   \item Interpretado
   \item Síntaxis simple y cercana a los lenguajes modernos
   \item Tipado dinámico y débil
   \item Tipos de datos simples y compuestos
   \item Referencia de datos
   \item Funciones y operadores sobre los distintos tipos de datos
   \item Sentencias de control 
   \item Variable de ámbito global y local
   \item Definición de funciones
   \item Paso de parámetros por valor y por referencia
   \item Funciones de orden superior
   \item Clausura de funciones
   \item Funciones anónimas
   \item Aplicación parcial de funciones
   \item Decoradores
   \item Definición de clases de objetos
   \item Creación e instanciación de objetos
   \item Tipos de datos como clases de objetos
   \item Visibilidad de métodos y atributos
   \item Definición estática de métodos y atributos
   \item Polimorfismo
   \item Duck typing
   \item Herencia simple
   \item Métodos mágicos
   \item Dynamic binding
   \item Excepciones
   \item Evaluación por cortocircuito devoliendo último valor
   \item Operadores condicionales 
   \item Funciones de fechas y tiempo \\
   \item Funciones de creación y acceso a ficheros
   \item Concurrente
   \item Recolector de basura
   \end{itemize}
\item Web OMI project.
\end{itemize}


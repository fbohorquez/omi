% ------------------------------------------------------------------------------
% Este fichero es parte de la plantilla LaTeX para la realización de Proyectos
% Final de Grado, protegido bajo los términos de la licencia GFDL.
% Para más información, la licencia completa viene incluida en el
% fichero fdl-1.3.tex

% Copyright (C) 2012 SPI-FM. Universidad de Cádiz
% ------------------------------------------------------------------------------

En este capítulo se presenta el plan de pruebas del sistema de información, incluyendo los diferentes tipos de pruebas que se han llevado a cabo, ya sean manuales (mediante listas de comprobación) o automatizadas mediante algún software específico de pruebas.

\section{Estrategia}
En esta sección se debe incluir el alcance de las pruebas, hasta donde se pretende llegar con ellas, si se registrarán todas o sólo aquellas de un cierto tipo y cómo se interpretarán y evaluarán los resultados.
También, se incluirá el procedimiento a seguir para las pruebas de regresión, esto es, la repetición de ciertas pruebas para comprobar que nuevos cambios que se vayan introduciendo no originen errores en el software ya probado.

\section{Entorno de Pruebas}
Incluir en este apartado los requisitos de los entornos hardware/software donde se ejecutarán las pruebas.

\section{Roles}
Describir en esa sección cuáles serán los perfiles y participantes necesarios para la ejecución de cada uno de los niveles de prueba.

\section{Niveles de Pruebas}
En este sección se documentan los diferentes tipos de pruebas que se han llevado a cabo, ya sean manuales o automatizadas mediante algún software específico de pruebas.

\subsection{Pruebas Unitarias}
Las pruebas unitarias tienen por objetivo localizar errores en cada nuevo artefacto software desarrollado, antes que se produzca la integración con el resto de artefactos del sistema.

\subsection{Pruebas de Integración}
Este tipo de pruebas tienen por objetivo localizar errores en módulos o subsistemas completos, analizando la interacción entre varios artefactos software.

\subsection{Pruebas de Sistema}
En esta actividad se realizan las pruebas de sistema de modo que se asegure que el sistema cumple con todos los requisitos establecidos: funcionales, de almacenamiento, reglas de negocio y no funcionales. Se suelen desarrollar en un entorno específico para pruebas.

\subsubsection{Pruebas Funcionales}
Con estas pruebas se analiza el buen funcionamiento de la implementación de los flujos normales y alternativos de los distintos casos de uso del sistema.

\subsubsection{Pruebas No Funcionales}
Estas pruebas pretenden comprobar el funcionamiento del sistema, con respecto a los requisitos no funcionales identificados: eficiencia, seguridad, etc.

\subsection{Pruebas de Aceptación}
El objetivo de estas pruebas es demostrar que el producto está listo para el paso a producción. Suelen ser las mismas pruebas que se realizaron anteriormente pero en el entorno de producción. En estas pruebas, es importante la participación del cliente final.
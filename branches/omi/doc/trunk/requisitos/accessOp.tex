\subsection{Operadores de acceso}
\subsubsection{Acceso a variable}
\begin{framed}
	\begin{description}
		\item [Número:] \cn
		\item [Nombre:] Acceso a variable.
		\item [Categoría:] Operadores de acceso.
		\item [Descripción:] Se hace necesario la gestión de los símbolos variables creados durante la ejecución, lo que implica el
		acceso a los datos referenciados por estos. Se deberá poder acceder a los datos referenciados por una variable
		simplemente mediante una expresión formada por el identificador asociado a la misma. El acceso a una variable debe originar un
		valor por defecto si esta no tiene un valor asociado. El valor por defecto dependerá del contexto:  
		\begin{description}
			\item[Clave de array:] Si la variable se utiliza para la clave de un array, o cualquier tipo
			derivado de este, el valor por defecto de esta es una cadena de caracteres que representa
			el identificador con el que se accede a la misma.
			\item [Cualquier otro:] En cualquier otro contexto el valor por defecto será el valor nulo.
		\end{description}
		Una operación de acceso a variable representa una expresión que deberá tener, como valor asociado tras su ejecución, el valor al
		que referencia el símbolo variable.
	\end {description}
\end{framed}


\subsubsection{Acceso a símbolo de dato compuesto}
\begin{framed}
	\begin{description}
		\item [Número:] \cn
		\item [Nombre:] Acceso a símbolo de dato compuesto.
		\item [Categoría:] Operadores de acceso.
		\item [Descripción:] Dado a que existen tipos de datos compuestos que deben mantener su propia tabla de símbolos,
		se precisa de un mecanismo para acceder a estos a partir del propio dato. Este mecanismo deberá devolver el símbolo al que se
		accede, pudiéndose aplicar cualquiera de las operaciones sobre símbolos.
	\end {description}
\end{framed}

\subsubsection{Acceso a última posición}
\begin{framed}
	\begin{description}
		\item [Número:] \cn
		\item [Nombre:] Acceso a última posición.
		\item [Categoría:] Operadores de acceso.
		\item [Descripción:] La mayoría de datos compuestos consisten en un listado de elementos. Se hace necesario un mecanismo
		para referenciar el final de este listado. A esta referencia se le podrá asignar algún dato, lo que lo añadirá al final del listado.
	\end {description}
\end{framed}

\subsubsection{Acceso a variables de entorno del sistema}
\begin{framed}
	\begin{description}
		\item [Número:] \cn
		\item [Nombre:] Acceso a variables de entorno del sistema.
		\item [Categoría:] Operadores de acceso.
		\item [Descripción:] Se necesita de un mecanismo para acceder a las variables de entorno definidas en el sistema operativo. 
	\end{description}
\end{framed}


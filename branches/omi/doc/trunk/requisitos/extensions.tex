\subsection{Extensiones}
\subsubsection{Extensión}
\begin{framed}
	\begin{description}
		\item [Número:] \cn
		\item [Nombre:] Extensión.
		\item [Categoría:] Extensiones.
		\item [Descripción:] La funcionalidad y características del intérprete deben ser extensible mediante módulos dinámicos. 
      Estos módulos añadirán sentencias, operadores y demás elementos propios de un lenguaje de programación. 
      
      Para que un extensión pueda ser utilizada se deberá cargar. 
	\end{description}
\end{framed}

\subsubsection{Carga de extensiones mediante fichero de configuración}
\begin{framed}
	\begin{description}
		\item [Número:] \cn
		\item [Nombre:] Carga de extensiones mediante fichero de configuración.
		\item [Categoría:] Extensiones.
		\item [Descripción:] Se debe facilitar un mecanismo que permita especificar un listado de extensiones que serán cargados al ejecutarse
      el interprete. Estos módulos serán especificados en un fichero de texto plano separados mediante saltos de línea. Toda ejecución del interprete
      conllevará la carga de las extensiones especificadas.
	\end{description}
\end{framed}

\subsubsection{Carga de extensiones mediante sentencia}
\begin{framed}
	\begin{description}
		\item [Número:] \cn
		\item [Nombre:] Carga de extensiones mediante sentencia.
		\item [Categoría:] Extensiones.
		\item [Descripción:] Se debe facilitar un mecanismo que permita cargar una extensión en tiempo de ejecución. Para ello se deberá
      facilitar la ruta del módulo correspondiente a la extensión como una cadena de caracteres. Tras la carga de la extensión las 
      características de esta serán añadidas al interprete.
	\end{description}
\end{framed}

\subsubsection{Biblioteca GNU de internacionalización (gettext)}
\begin{framed}
	\begin{description}
		\item [Número:] \cn
		\item [Nombre:] Extensión gettext.
		\item [Categoría:] Extensión gettext.
		\item [Descripción:] Se deberá facilitar a modo de extensión la funcionalidad y características de la 
      biblioteca GNU de internacionalización (i18n), gettext. 
	\end{description}
\end{framed}

\subsubsection{Operaciones sobre un SGBD Mysql}
\begin{framed}
	\begin{description}
		\item [Número:] \cn
		\item [Nombre:] Extensión mySQL.
		\item [Categoría:] Extensión mySQL.
		\item [Descripción:] Se deberá facilitar una extensión que amplíe las capacidades del interprete mediante 
      sentencias y recursos que permitan la interacción con un sistema de gestión de base de datos mySQL.
	\end{description}
\end{framed}

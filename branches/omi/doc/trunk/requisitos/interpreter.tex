


\subsection{Interprete}

\subsubsection{Léxico}
\begin{framed}
	\begin{description}
		\item [Número:] \cn
		\item [Nombre:] Léxico.
		\item [Categoría:] Interprete.
		\item [Descripción:] El sistema debe fijar el léxico del lenguaje conformado por una conjunto de palabras y expresiones bien definidas y acotadas.
	\end{description}
\end{framed}

\subsubsection{Gramática}
\begin{framed}
	\begin{description}
		\item [Número:] \cn
		\item [Nombre:] Gramática.
		\item [Categoría:] Interprete.
		\item [Descripción:] El sistema debe definir una gramática que representará el lenguaje. La gramática debe ser libre de contexto, clara
		y uniforme en toda su extensión. Además debe estar libre de ambigüedades.
	\end{description}
\end{framed}

\subsubsection{Semántica}
\begin{framed}
	\begin{description}
		\item [Número:] \cn
		\item [Nombre:] Interpretación semántica.
		\item [Categoría:] Interprete.
		\item [Descripción:] Dado un contenido fuente el sistema debe analizarlo en función al léxico (análisis léxico) y la gramática (análisis sintáctico)
		del lenguaje y producir el resultado semántico asociado.
	\end {description}
\end{framed}

\subsubsection{Comentarios}
\begin{framed}
	\begin{description}
		\item [Número:] \cn
		\item [Nombre:] Comentarios.
		\item [Categoría:] Interprete.
		\item [Descripción:] Se ha de contemplar un mecanismo para añadir comentarios al contenido fuente que serán ignorados
		durante la tarea de interpretación. 
      
      Los comentarios comprenderán desde un carácter ``\#'', o bien ``//'', hasta fin de línea.
      
      Por otro lado se ha de contemplar los comentarios de múltiples líneas, que deberán estar contenidos entre ``/*'' y ``*/''.
	\end{description}
\end{framed}

\subsubsection{Contenido fuente}
\begin{framed}
	\begin{description}
		\item [Número:] \cn
		\item [Nombre:] Fuente desde línea de comandos.
		\item [Categoría:] Interprete.
		\item [Descripción:] El interprete debe ser capaz de obtener contenido fuente desde una línea de comandos.
	\end {description}
\end{framed}

\begin{framed}
	\begin{description}
		\item [Número:] \cn
		\item [Nombre:] Fuente desde entrada estándar.
		\item [Categoría:] Interprete.
		\item [Descripción:] El interprete debe ser capaz de obtener contenido fuente desde la entrada estándar del sistema.
	\end {description}
\end{framed}

\begin{framed}
	\begin{description}
		\item [Número:] \cn
		\item [Nombre:] Fuente desde fichero.
		\item [Categoría:] Interprete.
		\item [Descripción:] El interprete debe ser capaz de obtener contenido fuente desde un fichero.
	\end {description}
\end{framed}

\subsubsection{Entorno}
\begin{framed}
	\begin{description}
		\item [Número:] \cn
		\item [Nombre:] Entorno de ejecución.
		\item [Categoría:] Interprete.
		\item [Descripción:] El interprete debe definir un entorno de ejecución en
		el que se controlen parámetros de entrada, variables de entornos del sistema operativo e información sobre
		el proceso como número de línea actual y los errores producidos.
	\end {description}
\end{framed}

\subsubsection {Parámetros}
\begin{framed}
	\begin{description}
		\item [Número:] \cn
		\item [Nombre:] Parámetros al programa.
		\item [Categoría:] Entrada.
		\item [Descripción:] Se debe facilitar un mecanismo para que el contenido fuente pueda recibir parámetros de entrada desde la
		invocación a su interpretación. Estos parámetros deberán ser copiados a símbolos variables accesibles desde el contenido fuente. 
		Adicionalmente se tratará otro parámetro que se corresponderá con el número de parámetros dados.
	\end{description}
\end{framed}

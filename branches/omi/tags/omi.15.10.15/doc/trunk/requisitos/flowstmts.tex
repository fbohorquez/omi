\subsection{Sentencias de control de flujo}
\subsubsection{Inclusión de ficheros}
\begin{framed}
	\begin{description}
		\item [Número:] \cn
		\item [Nombre:] Inclusión de ficheros.
		\item [Categoría:] Sentencias de control de flujo.
		\item [Descripción:] Se ha de facilitar un mecanismo para incluir en un punto de la ejecución contenido fuente localizado en recurso
		externo. El recurso consistirá en un fichero con sentencias interpretables.
	\end {description}
\end{framed}

\subsubsection{Saltar a etiqueta}
\begin{framed}
	\begin{description}
		\item [Número:] \cn
		\item [Nombre:] Saltar a etiqueta.
		\item [Categoría:] Sentencias de control de flujo.
		\item [Descripción:] Se ha de facilitar un mecanismo para llevar el flujo de ejecución a la sentencia  
		referenciada por una etiqueta.
	\end {description}
\end{framed}

\subsubsection{Sentencia if} 
\begin{framed}
	\begin{description}
		\item [Número:] \cn
		\item [Nombre:] Sentencia if.
		\item [Categoría:] Sentencias de control de flujo.
		\item [Descripción:] Deben de existir una serie de sentencias condicionales que alteren el flujo de ejecución. Las  
		sentencias if deberán estar construidas por bloques de sentencias y una serie de expresiones denominadas ``condiciones''.
		La interpretación de una sentencia de este tipo debe consistir en la evaluación lógica de las ``condiciones'' para determinar el
		bloque de sentencias que se ejecutará. Las formas de la sentencia if que el interprete debe aceptar son las siguientes:
		\begin{itemize}
			\item if (cond) stmts
			\item if (cond) stmts else stmts
			\item if (cond) stmts elif (cond) stmts ... else stmts
		\end{itemize}
	\end {description}
\end{framed}

\subsubsection{Sentencia switch}
\begin{framed}
	\begin{description}
		\item [Número:] \cn
		\item [Nombre:] Sentencia switch.
		\item [Categoría:] Sentencias de control de flujo.
		\item [Descripción:] El interprete debe ser capaz de interpretar sentencias del tipo switch case. Estas
		constan de una lista de bloques de sentencias precedidas de una expresión denominada ``caso''. Dada una expresión base
		esta debe ser comparada mediante la operación de igualdad con cada uno de los casos, ejecutando el bloque correspondiente al ``caso''
		cuya comparación sea positiva y todos los bloques siguientes. Se deberá poder especificar un bloque denominado ``default''
		que no dispondrá de expresión ``caso'' y será ejecutado sin aplicar condición alguna.
	\end {description}
\end{framed}

\subsubsection{Sentencia while}
\begin{framed}
	\begin{description}
		\item [Número:] \cn
		\item [Nombre:] Sentencia while.
		\item [Categoría:] Sentencias de control de flujo.
		\item [Descripción:] El interprete debe ser capaz de interpretar sentencias del tipo while. Esta es una sentencia de control
		iterativa que consta de una expresión denominada ``condición'' y un bloque de sentencias. El bloque de sentencias debe ser ejecutado
	    mientras que ``condición'' permanezca verdadera.
	\end {description}
\end{framed}

\subsubsection{Sentencia do...while}
\begin{framed}
	\begin{description}
		\item [Número:] \cn
		\item [Nombre:] Sentencia do...while.
		\item [Categoría:] Sentencias de control de flujo.
		\item [Descripción:] El interprete debe ser capaz de interpretar sentencias del tipo do while. Esta es una sentencia de control
		iterativa que consta de una expresión denominada ``condición'' y un bloque de sentencias. El bloque de sentencias debe ser ejecutado
	    mientras que ``condición'' permanezca verdadera, llevándose a cabo la ejecución al menos una vez.
	\end {description}
\end{framed}

\subsubsection{Sentencia for}
\begin{framed}
	\begin{description}
		\item [Número:] \cn
		\item [Nombre:] Sentencia for.
		\item [Categoría:] Sentencias de control de flujo.
		\item [Descripción:] El interprete debe ser capaz de interpretar sentencias del tipo for. Esta es una sentencia de control
		iterativa que consta de tres expresiones denominadas ``inicialización'', ``condición'' y ``paso'', además de un bloque de sentencias.
		Primero se ha de evaluar la expresión ``inicialización'', luego el bloque de sentencias se ejecutará mientras ``condición'' se
		valore como verdadera. La expresión ``paso'' se deberá ejecutar al finalizar cada iteración.   %GENERADORES DE EXPRESIONES
	\end {description}
\end{framed}

\subsubsection{Sentencia foreach}
\begin{framed}
	\begin{description}
		\item [Número:] \cn
		\item [Nombre:] Sentencia foreach.
		\item [Categoría:] Sentencia de control de flujo.
		\item [Descripción:] Se han de de interpretar sentencias del tipo forearch. Esta es una sentencia de control
		iterativa que consta un bloque de sentencias, de una expresión denominada ``conjunto'' y  un identificador denominado ``valor''.
		Se debe poder, aunque de forma opcional, especificar otro identificador que se denominará ``clave''.
		El bloque de sentencias será ejecutado de forma iterativa en función el tipo de dato y valor de ``conjunto''.  El ``conjunto''
		será evaluado para determinar el número de iteraciones y el valor que se le asignará como variables a los identificadores en
		cada iteración. Dependiendo del tipo de la expresión ``conjunto'' la sentencia foreach deberá actuar como sigue:
		
		\begin{description}
			\item [Tipo lógico:] El bloque de sentencias se ejecutará mientras ``conjunto'' sea verdadero. El identificador ``valor'' tomará
			el valor verdadero. En el caso en el que se especifique un identificador ``clave'' a este no se le asignará ningún valor.
			\item [Tipo aritmético:] Si ``conjunto'' representa un número mayor que cero el bloque de sentencias se ejecutará
			tantas veces como el valor numérico que representa. En cada iteración ``valor'' se le asignará el número de la iteración
			comenzando por cero. Si se presenta un identificador ``clave'' a este no se le asignará ningún valor.
			Si el valor de ``conjunto'' es menor o igual a cero el bloque no deberá ejecutarse.
			\item [Tipo cadena de caracteres:] Si ``conjunto'' es una cadena de caracteres que representa un número racional la
			ejecución deberá ser como si de un tipo aritmético se tratase. Si el ``conjunto'' es una cadena que no representa un número racional
			el bloque de sentencias se ejecutará por cada carácter en la cadena. En este último caso a ``valor'' se le asignará el carácter
			contemplado en cada iteración. Si se dio un identificador ``clave'' este no será asignado.
			\item [Tipo array:] Si ``conjunto'' es un array, u otro tipo de dato derivado de este como un objeto, el bloque de sentencias
			se ejecuta por cada elemento en el mismo. Al identificador ``valor'' se le asignará el valor del elemento en el array correspondiente
			a la iteración. En el caso de que se facilite un identificador ``clave'' este deberá tomar la clave del elemento en el array.
			\item [Otros tipos:] No se llevará a cabo ninguna operación.
		\end{description}
	\end {description}
\end{framed}

\subsubsection{Sentención de iteración ágil}
\begin{framed}
	\begin{description}
		\item [Número:] \cn
		\item [Nombre:] Sentencia de iteración ágil.
		\item [Categoría:] Sentencia de control de flujo.
		\item [Descripción:] Se ha de facilitar una sentencia de control que permita
		iterar un bloque de sentencias en función una expresión ``conjunto'' de forma ágil y sencilla.
		Para ello esta sentencia deberá operar igual que la sentencia foreach pero sin ser necesario, aunque posible,
		dar un identificador ``valor'' sobre el que se realizará la asignación. En lugar de ello la asignación que se produce
		en cada iteración se deberá realizar sobre un símbolo con identificador fijo y contenido variable denominado iterador.
		El iterador debe ser accesible desde el bloque de sentencias. Además se debe contemplar el acceso al iterador de 
      varias sentencias de ciclo ágil cuando estas se presentan de forma anidada. 
	\end {description}
\end{framed}

\subsubsection{Sentencia with}
\begin{framed}
	\begin{description}
		\item [Número:] \cn
		\item [Nombre:] Sentencia with.
		\item [Categoría:] Sentencia de control de flujo.
		\item [Descripción:] Se deberá facilitar un mecanismo que permita establecer una estrucutra compuesta como contexto. Así todo acceso que se
      realice, y cuya definición no exista, se deberá hacer sobre los elementos que de la estructura compuesta utilizada como contexto. 
      Esta sentencia se deberá construir a partir del dato compuesto y un bloque de sentencias sobre el que se aplicará el contexto.
	\end{description}
\end{framed}

\subsubsection{Finalizar bloque de sentencias}
\begin{framed}
	\begin{description}
		\item [Número:] \cn
		\item [Nombre:] Finalizar bloque de sentencias.
		\item [Categoría:] Sentencias de control de flujo.
		\item [Descripción:] Se ha de disponer de un mecanismo para indicar que el flujo debe salir de una sentencia de control. Se ha de contemplar
		las sentencias anidadas.
	\end {description}
\end{framed}



\subsubsection{Finalizar iteración}
\begin{framed}
	\begin{description}
		\item [Número:] \cn
		\item [Nombre:] Finalizar iteración.
		\item [Categoría:] Sentencias de control de flujo.
		\item [Descripción:] El sistema debe facilitar algún recurso que permita finalizar la iteración actual
		de una sentencia de control en ejecución y comenzar con la siguiente. Este
		mecanismo debe contemplar la posibilidad de salir de varias sentencias de control anidadas.
	\end {description}
\end{framed}

\subsubsection{Finalizar ejecución}
\begin{framed}
	\begin{description}
		\item [Número:] \cn
		\item [Nombre:] Finalizar ejecución.
		\item [Categoría:] Sentencias de control de flujo.
		\item [Descripción:] Se ha de disponer de un mecanismo para que el sistema finalice de forma inmediata de
		interpretar el contenido fuente.
	\end {description}
\end{framed}

\subsubsection{Capturar excepción}

Aún por completar.

\subsubsection{Lanzar excepción}

Aún por completar.

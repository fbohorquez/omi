\subsection{Operadores aritméticos}

\subsubsection{Suma}
%\begin{framed}
	\begin{description}
		\item [Número:] \cn
		\item [Nombre:] Suma.
		\item [Categoría:] Operadores aritméticos.
		\item [Descripción:] Se debe contemplar la expresión correspondiente a la operación aritmética ``suma''. Para realizar
		esta operación se deberá tomar el valor aritmético de cada operando. Tras realizarse la operación, el valor de la expresión
		deberá ser el resultado aritmético de la misma.
	\end {description}
%\end{framed}

\subsubsection{Diferencia}
%\begin{framed}
	\begin{description}
		\item [Número:] \cn
		\item [Nombre:] Diferencia.
		\item [Categoría:] Operadores aritméticos.
		\item [Descripción:] Se debe contemplar la expresión correspondiente a la operación aritmética ``resta''. Para realizar
		esta operación se deberá tomar el valor aritmético de cada operando. Tras realizarse la operación, el valor de la expresión
		deberá ser el resultado aritmético de la misma.
	\end {description}
%\end{framed}

\subsubsection{Producto}
%\begin{framed}
	\begin{description}
		\item [Número:] \cn
		\item [Nombre:] Producto.
		\item [Categoría:] Operadores aritméticos.
		\item [Descripción:] Se debe contemplar la expresión correspondiente a la operación aritmética ``producto''. Para realizar
		esta operación se deberá tomar el valor aritmético de cada operando. Tras realizarse la operación, el valor de la expresión
		deberá ser el resultado aritmético de la misma.
	\end {description}
%\end{framed}

\subsubsection{División}
%\begin{framed}
	\begin{description}
		\item [Número:] \cn
		\item [Nombre:] División.
		\item [Categoría:] Operadores aritméticos.
		\item [Descripción:] Se debe contemplar la expresión correspondiente a la operación aritmética ``división''. Para realizar
		esta operación se deberá tomar el valor aritmético de cada operando. El segundo operando debe ser
		distinto de 0. Si el segundo operando tiene valor aritmético 0 se deberá mostrar un error que informe del caso.
		Tras realizarse la operación, el valor de la expresión
		deberá ser el resultado aritmético de la misma.
	\end {description}
%\end{framed}

\subsubsection{Potencia}
%\begin{framed}
	\begin{description}
		\item [Número:] \cn
		\item [Nombre:] Potencia.
		\item [Categoría:] Operadores aritméticos.
		\item [Descripción:] Se debe contemplar la expresión correspondiente a la operación aritmética ``potencia''. Para realizar
		esta operación se deberá tomar el valor aritmético de cada operando. Tras realizarse la operación, el valor de la expresión
		deberá ser el resultado aritmético de la misma.
	\end {description}
%\end{framed}

\subsubsection{Módulo}
%\begin{framed}
	\begin{description}
		\item [Número:] \cn
		\item [Nombre:] Módulo.
		\item [Categoría:] Operadores aritméticos.
		\item [Descripción:] Se debe contemplar la expresión correspondiente a la operación aritmética ``módulo''. Para realizar
		esta operación se deberá tomar el valor aritmético de cada operando. El segundo operando debe ser
		distinto de 0. Si el segundo operando tiene valor aritmético 0 se deberá mostrar un error que informe del caso.
		Tras realizarse la operación, el valor de la expresión deberá ser el resultado aritmético de la misma.
	\end {description}
%\end{framed}

\subsubsection{Tamaño}
%\begin{framed}
	\begin{description}
		\item [Número:] \cn
		\item [Nombre:] Tamaño.
		\item [Categoría:] Operadores aritméticos.
		\item [Descripción:] Se precisa de algún mecanismo que dado un dato calcule el tamaño de este. Este operador calculará el
		tamaño dependiendo del tipo de dato del operando. Tras ejecutase la expresión el valor que tome será de tipo aritmético.
		\begin{description}
			\item[Lógico:] Si es verdadero el tamaño es uno, si es falso será cero.
			\item[Aritmético:] Tomará el número de dígitos decimales.
			\item[Cadena:] El tamaño será el número de caracteres de la cadena.
			\item[Array:] Para el tipo de dato array u otros derivados se el tamaño será el número de elementos contenidos en el mismo.
			\item[Otro tipo de dato:] Se deberá dar un error de tipos.
		\end{description}
	\end {description}
%\end{framed}

\subsubsection{Incremento y asignación}
%\begin{framed}
	\begin{description}
		\item [Número:] \cn
		\item [Nombre:] Incremento y asignación.
		\item [Categoría:] Operadores aritméticos.
		\item [Descripción:] Dado un identificador o expresión que referencia a una dato variable, el valor de esta se debe poder incrementar y
		reasignar. Para ello se tomará el valor aritmético de la variable, se incrementará en una unidad y se reasignará a la misma.
		El valor de la expresión será el de la variable incrementada.
	\end {description}
%\end{framed}

\subsubsection{Asignación e incremento}
%\begin{framed}
	\begin{description}
		\item [Número:] \cn
		\item [Nombre:] Asignación e incremento.
		\item [Categoría:] Operadores aritméticos.
		\item [Descripción:] Dado un identificador o expresión que referencia a una dato variable, el valor de esta se debe poder incrementar y
		reasignar. Para ello se tomará el valor aritmético de la variable, se incrementará en una unidad y se reasignará a la misma.
		El valor de la expresión será el de la variable antes de ser incrementada.
	\end {description}
%\end{framed}

\subsubsection{Decremento y asignación}
%\begin{framed}
	\begin{description}
		\item [Número:] \cn
		\item [Nombre:] Decremento y asignación.
		\item [Categoría:] Operadores aritméticos.
		\item [Descripción:] Dado un identificador o expresión que referencia a una dato variable, el valor de esta se debe poder decrementar y
		reasignar. Para ello se tomará el valor aritmético de la variable, se decrementará en una unidad y se reasignará a la misma.
		El valor de la expresión será el de la variable decrementada.
	\end {description}
%\end{framed}

\subsubsection{Asignación y decremento}
%\begin{framed}
	\begin{description}
		\item [Número:] \cn
		\item [Nombre:] Asignación y decremento.
		\item [Categoría:] Operadores aritméticos.
		\item [Descripción:] Dado un identificador o expresión que referencia a una dato variable, el valor de esta se debe poder decrementar y
		reasignar. Para ello se tomará el valor aritmético de la variable, se decrementará en una unidad y se reasignará a la misma.
		El valor de la expresión será el de la variable antes de ser decrementada.
	\end {description}
%\end{framed}

\subsubsection{Suma y asignación}
%\begin{framed}
	\begin{description}
		\item [Número:] \cn
		\item [Nombre:] Suma y asignación.
		\item [Categoría:] Operadores aritméticos.
		\item [Descripción:] Dado un identificador o expresión que referencia a una dato variable, al valor de esta se ha de poder sumar otra expresión
		y reasignarle el resultado. Para ello se tomará el valor aritmético de la variable, se le sumará el valor aritmético de la expresión
		y se reasignará a la variable el resultado. El valor de la expresión será el resultado de la suma aritmética.
	\end {description}
%\end{framed}

\subsubsection{Diferencia y asignación}
%\begin{framed}
	\begin{description}
		\item [Número:] \cn
		\item [Nombre:] Diferencia y asignación.
		\item [Categoría:] Operadores aritméticos.
		\item [Descripción:] Dado un identificador o expresión que referencia a una dato variable, al valor de esta se ha de poder restar otra expresión
		y reasignarle el resultado. Para ello se tomará el valor aritmético de la variable, se le restará el valor aritmético de la expresión
		y se reasignará a la variable el resultado. El valor de la expresión será el resultado de la resta aritmética.
	\end {description}
%\end{framed}

\subsubsection{Producto y asignación}
%\begin{framed}
	\begin{description}
		\item [Número:] \cn
		\item [Nombre:] Producto y asignación.
		\item [Categoría:] Operadores aritméticos.
		\item [Descripción:] Dado un identificador o expresión que referencia a una dato variable, al valor de esta se ha de poder multiplicar
		otra expresión y reasignarle el resultado. Para ello se tomará el valor aritmético de la variable, se calculará el producto  y se reasignará a la variable el resultado. El valor de la expresión será el resultado del producto aritmético.
	\end {description}
%\end{framed}

\subsubsection{División y asignación}
%\begin{framed}
	\begin{description}
		\item [Número:] \cn
		\item [Nombre:] División y asignación.
		\item [Categoría:] Operadores aritméticos.
		\item [Descripción:] Dado un identificador o expresión que referencia a una dato variable, al valor de esta se ha de poder dividir por
		otra expresión y reasignarle el resultado. Para ello se tomará el valor aritmético de la variable, se realizará la división y se reasignará a la variable el resultado. La expresión no puede tener un valor aritmético de cero.
		El valor de la expresión será el resultado de la división aritmética.
	\end {description}
%\end{framed}

\subsubsection{Potencia y asignación}
%\begin{framed}
	\begin{description}
		\item [Número:] \cn
		\item [Nombre:] Potencia y asignación.
		\item [Categoría:] Operadores aritméticos.
		\item [Descripción:] Dado un identificador o expresión que referencia a una dato variable, al valor de esta se ha de poder elevar a
		otra expresión y reasignarle el resultado. Para ello se tomará el valor aritmético de la variable, se elevará  y se reasignará a la variable el resultado. El valor de la expresión será el resultado de la potencia.
	\end {description}
%\end{framed}

\subsubsection{Módulo y asignación}
%\begin{framed}
	\begin{description}
		\item [Número:] \cn
		\item [Nombre:] Módulo y asignación.
		\item [Categoría:] Operadores aritméticos.
		\item [Descripción:] Dado un identificador o expresión que referencia a una dato variable, al valor de esta se ha de poder dividir por
		otra expresión y reasignarle el resto originado. Para ello se tomará el valor aritmético de la variable, se realizará la división por el
		valor aritmético de la expresión y se reasignará a la variable el resto obtenido. La expresión no puede tener un valor aritmético de cero.
		El valor de la expresión será el resultado de la operación módulo.
	\end {description}
%\end{framed}

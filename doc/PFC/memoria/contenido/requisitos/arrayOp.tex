\subsection{Operadores sobre array}

\subsubsection{Dividir en fragmentos}
%\begin{framed}
	\begin{description}
		\item [Número:] \cn
		\item [Nombre:] Función dividir array en fragmentos.
		\item [Categoría:] Operadores sobre array.
		\item [Descripción:] Se precisa de un mecanismo que, dado un array y un valor aritmético. Se divida el array en varios subarray de tantos
      elementos como indique el valor (a excepción del último). 
      La expresión tomará como valor un array que contiene cada uno de los subarray.
	\end {description}
%\end{framed}

\subsubsection{Reducir mediante función}
%\begin{framed}
	\begin{description}
		\item [Número:] \cn
		\item [Nombre:] Función reducir array.
		\item [Categoría:] Operadores sobre array.
		\item [Descripción:] Se necesita de un mecanismo que dado un array y una función reduzca el array a un solo valor.
      La función de reducción deberá recibir como parámetro dos valores correspondiente al valor acumulado y al nuevo valor. 
      La función de reducción se ejecutará por cada elemento del array (excepto para el primero) tomando el valor acumulado y el nuevo valor, y devolviendo
      el próximo valor acumulado. Como valor este operador tomará el valor de la reducción. 
	\end {description}
%\end{framed}


\subsubsection{Obtener último}
%\begin{framed}
	\begin{description}
		\item [Número:] \cn
		\item [Nombre:] Función obtener último elemento de array.
		\item [Categoría:] Operadores sobre array.
		\item [Descripción:] Se precisa de un mecanismo que, dado un array se obtenga el 
último elemento de este, o el valor nulo si este está vacío.
	\end {description}
%\end{framed}

\subsubsection{Obtener primero}
%\begin{framed}
	\begin{description}
		\item [Número:] \cn
		\item [Nombre:] Función obtener primer elemento de array.
		\item [Categoría:] Operadores sobre array.
		\item [Descripción:] Se precisa de un mecanismo que, dado un array se obtenga el 
primer elemento de este, o el valor nulo si este está vacío.
	\end {description}
%\end{framed}

\subsubsection{Insertar en posición}
%\begin{framed}
	\begin{description}
		\item [Número:] \cn
		\item [Nombre:] Función para insertar un elemento en una posición de  array.
		\item [Categoría:] Operadores sobre array.
		\item [Descripción:] Se precisa de un mecanismo que, dado un array, un elemento
y una expresión aritmética que hace de posición, se inserte el elemento en dicha posición del array.
Si la posición se encuentra fuera de rango se dará un error de acceso.
	\end {description}
%\end{framed}

\subsubsection{Eliminar posición}
%\begin{framed}
	\begin{description}
		\item [Número:] \cn
		\item [Nombre:] Función eliminar elemento de array de una posición.
		\item [Categoría:] Operadores sobre array.
		\item [Descripción:] Se precisa de un mecanismo que, dado un array y una expresión aritmética que hace de posición, se elimine el elemento que ocupa dicha posición dentro del array.
Si la posición se encuentra fuera de rango se dará un error de acceso.
	\end {description}
%\end{framed}

\subsubsection{Insertar al inicio}
%\begin{framed}
	\begin{description}
		\item [Número:] \cn
		\item [Nombre:] Función insertar al inicio de un array.
		\item [Categoría:] Operadores sobre array.
		\item [Descripción:] Se precisa de un mecanismo que, dado un array y un elemento de un tipo arbitrario, se inserte el elemento al comienzo del array.
	\end {description}
%\end{framed}

\subsubsection{Insertar al final}
%\begin{framed}
	\begin{description}
		\item [Número:] \cn
		\item [Nombre:] Función insertar al final de un array.
		\item [Categoría:] Operadores sobre array.
		\item [Descripción:] Se precisa de un mecanismo que, dado un array y un elemento de un tipo arbitrario, se inserte el elemento al final del array.
	\end {description}
%\end{framed}

\subsubsection{Eliminar del inicio}
%\begin{framed}
	\begin{description}
		\item [Número:] \cn
		\item [Nombre:] Función eliminar del comienzo de un array.
		\item [Categoría:] Operadores sobre array.
		\item [Descripción:] Se precisa de un mecanismo que, dado un array, se elimine
el primer elemento del mismo. Como valor se tomará el elemento eliminado del array, o el valor nulo si este se encontraba vacío.
	\end {description}
%\end{framed}


\subsubsection{Eliminar del inicio}
%\begin{framed}
	\begin{description}
		\item [Número:] \cn
		\item [Nombre:] Función eliminar del final de un array.
		\item [Categoría:] Operadores sobre array.
		\item [Descripción:] Se precisa de un mecanismo que, dado un array, se elimine
el último elemento del mismo. Como valor se tomará el elemento eliminado del array, o el valor nulo si este se encontraba vacío.
	\end {description}
%\end{framed}

\subsection{Intérprete}

\subsubsection{Léxico, gramática y semántica}
%\begin{framed}
	\begin{description}
		\item [Número:] \cn
		\item [Nombre:] Léxico.
		\item [Categoría:] Intérprete.
		\item [Descripción:] El sistema debe fijar el léxico del lenguaje conformado por una conjunto de palabras y expresiones bien definidas y acotadas.
	\end{description}
%\end{framed}

%\begin{framed}
	\begin{description}
		\item [Número:] \cn
		\item [Nombre:] Gramática.
		\item [Categoría:] Intérprete.
		\item [Descripción:] El sistema debe definir una gramática que representará el lenguaje. La gramática debe ser libre de contexto, clara
		y uniforme en toda su extensión. Además debe estar libre de ambigüedades.
	\end{description}
%\end{framed}

%\begin{framed}
	\begin{description}
		\item [Número:] \cn
		\item [Nombre:] Interpretación semántica.
		\item [Categoría:] Intérprete.
		\item [Descripción:] Dado un contenido fuente el sistema debe analizarlo en función al léxico (análisis léxico) y la gramática (análisis sintáctico)
		del lenguaje y producir el resultado semántico asociado.
	\end {description}
%\end{framed}

\subsubsection{Contenido fuente}
%\begin{framed}
	\begin{description}
		\item [Número:] \cn
		\item [Nombre:] Comentarios.
		\item [Categoría:] Intérprete.
		\item [Descripción:] Se ha de contemplar un mecanismo para añadir comentarios al contenido fuente que serán ignorados
		durante la tarea de interpretación. 
      
      Los comentarios comprenderán desde un carácter ``\#'', o bien ``//'', hasta fin de línea.
      
      Por otro lado se ha de contemplar los comentarios de múltiples líneas, que deberán estar contenidos entre ``/*'' y ``*/''.
	\end{description}
%\end{framed}


%\begin{framed}
	\begin{description}
		\item [Número:] \cn
		\item [Nombre:] Fuente desde línea de comandos.
		\item [Categoría:] Intérprete.
		\item [Descripción:] El intérprete debe ser capaz de obtener contenido fuente desde una línea de comandos.
	\end {description}
%\end{framed}

%\begin{framed}
	\begin{description}
		\item [Número:] \cn
		\item [Nombre:] Fuente desde entrada estándar.
		\item [Categoría:] Intérprete.
		\item [Descripción:] El intérprete debe ser capaz de obtener contenido fuente desde la entrada estándar del sistema.
	\end {description}
%\end{framed}

%\begin{framed}
	\begin{description}
		\item [Número:] \cn
		\item [Nombre:] Fuente desde fichero.
		\item [Categoría:] Intérprete.
		\item [Descripción:] El intérprete debe ser capaz de obtener contenido fuente desde un fichero.
	\end {description}
%\end{framed}

%\begin{framed}
	\begin{description}
		\item [Número:] \cn
		\item [Nombre:] Fuente desde puerto TCP.
		\item [Categoría:] Intérprete.
		\item [Descripción:] El intérprete debe poder ser ejecutado en modo servidor, recibiendo una cadena de caracteres por un puerto TCP, e interpretándola. Esto ocasionará que el 
servidor devuelva por otro puerto una estructura de datos que representa el proceso de interpretación. 
	\end {description}
%\end{framed}

\subsubsection{Salida}

%\begin{framed}
	\begin{description}
		\item [Número:] \cn
		\item [Nombre:] Salida por la salida estándar.
		\item [Categoría:] Intérprete.
		\item [Descripción:] El intérprete debe poder utilizar la salida estándar del sistema
para imprimir el resultado de la interpretación según el código fuente. Para ello se deberá presentar
sentencias que permitan escribir en esta.
	\end {description}
%\end{framed}

%\begin{framed}
	\begin{description}
		\item [Número:] \cn
		\item [Nombre:] Salida estructurada.
		\item [Categoría:] Intérprete.
		\item [Descripción:] El intérprete debe de tener la capacidad de producir una serie 
de datos estructurados que representen el proceso de interpretación llevado a cabo para un determinado código fuente. Estos datos podrán ser almacenados en un fichero o devueltos por 
un puerto TCP.
	\end {description}
%\end{framed}

\subsubsection{Entorno}
%\begin{framed}
	\begin{description}
		\item [Número:] \cn
		\item [Nombre:] Entorno de ejecución.
		\item [Categoría:] Intérprete.
		\item [Descripción:] El intérprete debe definir un entorno de ejecución en
		el que se controlen parámetros de entrada, variables de entornos del sistema operativo e información sobre
		el proceso como número de línea actual y los errores producidos.
	\end {description}
%\end{framed}

\subsubsection {Parámetros}
%\begin{framed}
	\begin{description}
		\item [Número:] \cn
		\item [Nombre:] Parámetros al programa.
		\item [Categoría:] Entrada.
		\item [Descripción:] Se debe facilitar un mecanismo para que el contenido fuente pueda recibir parámetros de entrada desde la
		invocación a su interpretación. Estos parámetros deberán ser copiados a símbolos variables accesibles desde el contenido fuente. 
		Adicionalmente se tratará otro parámetro que se corresponderá con el número de parámetros dados.
	\end{description}
%\end{framed}

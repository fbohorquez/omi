\subsection{Operadores lógicos}
\subsubsection{AND lógico}
%\begin{framed}
	\begin{description}
		\item [Número:] \cn
		\item [Nombre:] AND lógico.
		\item [Categoría:] Operadores lógicos.
		\item [Descripción:] Se debe contemplar la expresión correspondiente a la operación lógica AND. Para ello se deberá tomar
		el valor lógico de cada uno de los operandos. La evaluación de la operación lógica AND debe ser de cortocircuito, tomándose el valor del
		último elemento evaluado. Así, aunque esta expresión se corresponde con un operador lógico, el valor de la misma será el del
		último elemento evaluado.
	\end {description}
%\end{framed}

\subsubsection{OR lógico}
%\begin{framed}
	\begin{description}
		\item [Número:] \cn
		\item [Nombre:] OR lógico.
		\item [Categoría:] Operadores lógicos.
		\item [Descripción:] Se debe contemplar la expresión correspondiente a la operación lógica OR. Para ello se deberá tomar
		el valor lógico de cada uno de los operandos. La evaluación de la operación lógica OR debe ser de cortocircuito, tomándose el valor del
		último elemento evaluado. Así, aunque esta expresión se corresponde con un operador lógico, el valor de la misma será el del
		último elemento evaluado.
	\end {description}
%\end{framed}

\subsubsection{NOT lógico}
%\begin{framed}
	\begin{description}
		\item [Número:] \cn
		\item [Nombre:] NOT lógico.
		\item [Categoría:] Operadores lógicos.
		\item [Descripción:] Se debe contemplar la expresión correspondiente a la operación lógica NOT. Para ello se deberá tomar
		el valor lógico de su único operando y negarlo. La expresión deberá tomar un valor de tipo booleano tras realizarse la operación.
	\end {description}
%\end{framed}

\subsubsection{Vacío}
%\begin{framed}
	\begin{description}
		\item [Número:] \cn
		\item [Nombre:] Vacío.
		\item [Categoría:] Operadores lógicos.
		\item [Descripción:] Se necesita de un operador que determine si un dato se considera vacío. Este operador tendrá un
		único operando y funcionará igual que el operador lógico NOT. El valor que tomará la expresión será lógico.
	\end {description}
%\end{framed}

\subsubsection{Es nulo}
%\begin{framed}
	\begin{description}
		\item [Número:] \cn
		\item [Nombre:] Es nulo.
		\item [Categoría:] Operadores lógicos.
		\item [Descripción:] Se necesita de un operador que determine si un dato o expresión contiene el valor nulo.
	\end {description}
%\end{framed}

\subsubsection{Menor que}
%\begin{framed}
	\begin{description}
		\item [Número:] \cn
		\item [Nombre:] Menor que.
		\item [Categoría:] Operadores de comparación.
		\item [Descripción:] Se debe contemplar la expresión correspondiente a la operación lógica ``menor que''. Para ello se deberá tomar
		el valor aritmético de cada operando. La expresión deberá tomar un valor de tipo booleano tras realizarse la operación.
	\end {description}
%\end{framed}

\subsubsection{Menor o igual que}
%\begin{framed}
	\begin{description}
		\item [Número:] \cn
		\item [Nombre:] Menor o igual que.
		\item [Categoría:] Operadores de comparación.
		\item [Descripción:] Se debe contemplar la expresión correspondiente a la operación lógica ``menor o igual que''. Para ello se deberá tomar
		el valor aritmético de cada operando. La expresión deberá tomar un valor de tipo booleano tras realizarse la operación.
	\end {description}
%\end{framed}

\subsubsection{Mayor que}
%\begin{framed}
	\begin{description}
		\item [Número:] \cn
		\item [Nombre:] Mayor que.
		\item [Categoría:] Operadores de comparación.
		\item [Descripción:] Se debe contemplar la expresión correspondiente a la operación lógica ``mayor que''. Para ello se deberá tomar
		el valor aritmético de cada operando. La expresión deberá tomar un valor de tipo booleano tras realizarse la operación.
	\end {description}
%\end{framed}

\subsubsection{Mayor o igual que}
%\begin{framed}
	\begin{description}
		\item [Número:] \cn
		\item [Nombre:] Mayor o igual que.
		\item [Categoría:] Operadores de comparación.
		\item [Descripción:] Se debe contemplar la expresión correspondiente a la operación lógica ``mayor o igual que''. Para ello se deberá tomar
		el valor aritmético de cada operando. La expresión deberá tomar un valor de tipo booleano tras realizarse la operación.
	\end {description}
%\end{framed}

\subsubsection{Igual que}
%\begin{framed}
	\begin{description}
		\item [Número:] \cn
		\item [Nombre:] Igual que.
		\item [Categoría:] Operadores de comparación.
		\item [Descripción:] Se debe contemplar la expresión correspondiente a la operación lógica ``igual que''. La operación de igualdad
		debe ser independiente de los tipos de datos de los operandos, aplicándose en función del tipo de dato más completo que compartan.
		Por ejemplo si se compara un dato cadena que no representa un número racional con uno aritmético, como el tipo de dato común a ambos
		es el booleano, ambos tomarán su valor lógico para la comparación.  
		Si ambos datos son tipos compuestos, se ha de comprobar mediante la operación de igualdad todos los elementos simples que lo componen
		por pares y de forma posicional.
		Como valor de la expresión se toma el valor booleano de la operación.
	\end {description}
%\end{framed}

\subsubsection{Idéntico que}
%\begin{framed}
	\begin{description}
		\item [Número:] \cn
		\item [Nombre:] Idéntico que.
		\item [Categoría:] Operadores de comparación.
		\item [Descripción:] Se debe contemplar la expresión correspondiente a la operación lógica ``idéntico que''. Esta operación
		se refiere a una operación lógica de igualdad pero contemplando además que los datos tengan el mismo tipo. Como valor
		de la expresión se debe tomar el valor booleano resultado de aplicar la operación.
	\end {description}
%\end{framed}

\subsubsection{Distinto que}
%\begin{framed}
	\begin{description}
		\item [Número:] \cn
		\item [Nombre:] Distinto que.
		\item [Categoría:] Operadores de comparación.
		\item [Descripción:] Se debe contemplar la expresión correspondiente a la operación lógica ``distinto que''. Esta operación
		debe ser independiente de los tipos de datos de los operandos, aplicándose en función del tipo de dato más completo que compartan.
		Por ejemplo si se compara un dato cadena que no representa un número racional con uno aritmético, como el valor más completo que
		ambos pueden tomar es el booleano, tomarán su valor lógico para la comparación.  
		Si ambos datos son tipos compuestos, se ha de comprobar mediante la operación de igualdad todos los elementos simples que lo componen
		por pares y de forma posicional.
		Como valor de la expresión se toma el valor booleano de la operación.
	\end {description}
%\end{framed}

\subsubsection{No idéntico que}
%\begin{framed}
	\begin{description}
		\item [Número:] \cn
		\item [Nombre:] No idéntico que.
		\item [Categoría:] Operadores de comparación.
		\item [Descripción:] Se debe contemplar la expresión correspondiente a la operación lógica ``no idéntico que''. Esta operación
		se corresponde con la operación inversa de la operación ``idéntico que''. Como valor
		de la expresión se debe tomar el valor booleano resultado de aplicar la operación.
	\end {description}
%\end{framed}

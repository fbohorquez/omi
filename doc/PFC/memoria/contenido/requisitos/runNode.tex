\subsection{Ejecución}
\subsubsection{Sentencias}
%\begin{framed}
	\begin{description}
		\item [Número:] \cn
		\item [Nombre:] Sentencia.
		\item [Categoría:] Ejecución.
		\item [Descripción:] Son las unidades interpretables más pequeña en las que se divide un contenido fuente. Las sentencias están sujetas a
		unas reglas sintácticas y encierran un significado semántico. El intérprete debe definir la gramática de cada sentencia y dotarlas de significado
		semántico. Toda sentencia debe finalizar con el carácter ``;'', excluyendo las sentencias formadas por bloques de sentencias. Aunque carezca
		de sentido práctico, para evitar posibles errores de codificación y mantener coherencia en la sintaxis y la definición del lenguaje, se debe
		contemplar la sentencia vacía que sólo conste del carácter ``;''.
	\end {description}
%\end{framed}

%\begin{framed}
	\begin{description}
		\item [Número:] \cn
		\item [Nombre:] Bloques de sentencias.
		\item [Categoría:] Ejecución.
		\item [Descripción:] Son un conjunto de sentencias que deberán ser interpretadas y ejecutadas secuencialmente. La disposición de
		sentencias en el bloque determinan el flujo de ejecución que se llevará a cabo cuando se intérprete el bloque. El contenido fuente
		en si mismo es un bloque de sentencias. Todo bloque de sentencias de más de una sentencia (con excepción del contenido fuente en si mismo)
		debe ir delimitado mediante llaves. Aunque no sea de uso común, para mantener coherencia en la sintaxis y la definición del lenguaje, se debe
		contemplar el bloque de sentencias vacío.
	\end {description}
%\end{framed}

\subsubsection{Expresiones}
%\begin{framed}
	\begin{description}
		\item [Número:] \cn
		\item [Nombre:] Expresiones.
		\item [Categoría:] Ejecución.
		\item [Descripción:] El intérprete debe ser capaz de evaluar expresiones. Estas son secuencias de datos, operadores, operandos,
		elementos de puntuación y/o palabras clave, que especifican una unidad computacional.
		Generalmente encierran un valor que se asocia a la expresión después de ser evaluada. Una sentencia puede estar formada por una o
		varias expresiones que deberán ser evaluadas o interpretadas para dotarla de significado. Una sentencia puede constar únicamente
		de una expresión en ese caso la sentencia es considerada la evaluación de dicha expresión.
		
		La expresión más simple equivale a un único dato, en este caso el valor de la expresión será el del dato.
	\end {description}
%\end{framed}

%\begin{framed}
	\begin{description}
		\item [Número:] \cn
		\item [Nombre:] Expresiones de tipo definido.
		\item [Categoría:] Ejecución.
		\item [Descripción:] Son expresiones cuyo valor es de un tipo definido y fijo. El sistema debe interpretar estas expresiones para determinar el valor
		asociado a la mismas en un momento dado.
	\end {description}
%\end{framed}

%\begin{framed}
	\begin{description}
		\item [Nombre:] Expresiones de tipo no definido.
		\item [Categoría:] Ejecución.
		\item [Número:] \cn
		\item [Descripción:] Son expresiones cuyo valor no tiene un tipo definido ni fijo, sino que es durante la interpretación cuando se determina el tipo.
		El sistema debe interpretar estas expresiones para determinar, además del valor asociado a la mismas, el tipo de dato que guardan en un momento
		dado.
	\end {description}
%\end{framed}

\subsubsection{Datos}
%\begin{framed}
	\begin{description}
		\item [Nombre:] Datos.
		\item [Categoría:] Ejecución.
		\item [Número:] \cn
		\item [Descripción:] El intérprete deberá operar sobre datos. El contenido fuente 
      definirá cómo se han de construir y/o acceder a los datos y las operaciones que se realizarán sobre ellos durante la ejecución. 
      
      Un dato será tratado en función de su tipo. El tipo de dato lo dota de una semántica, un significado.
      Así, todo dato deberá ser considerado un objeto, por lo que tendrán unas propiedades y funcionalidad ligadas al tipo como el que es tratado.
      
	\end {description}
%\end{framed}

\subsubsection{Operadores}
%\begin{framed}
	\begin{description}
		\item [Número:] \cn
		\item [Nombre:] Operadores.
		\item [Categoría:] Operadores.
		\item [Descripción:] Se ha de facilitar una serie de operadores que permitan manipular los datos. Los operadores
		son en si mismo expresiones, por los que estos tendrán un valor asociado tras ejecutarse. Los operadores constarán
		de una serie operandos que intervendrán en la operación y que serán a su vez otras expresiones.
		
      Los operadores se clasificarán en función del tipo de valor que tendrán tras la ejecución, los tipos de
		los operandos y/o la naturaleza del operador en si.  
      
      Un operador puede presentarse en forma de función, los operandos serán los parámetros de esta. La ejecución
      de la función conllevará la realización de la operación asociada al operador.
	\end {description}
%\end{framed}

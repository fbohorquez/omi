\subsection{Web del proyecto OMI}

\subsubsection{Sitio web}
%~ \begin{framed}
	\begin{description}
		\item [Número:] \cn
		\item [Nombre:] Sitio web.
		\item [Categoría:] Web OMI.
		\item [Descripción:] Se ha de disponer de una web que sirva de plataforma para acceder al 
      proyecto y todos los recursos que este brinda. Esta web tendrá una estructura estática.
	\end {description}
%~ \end{framed}

\subsubsection{Home del sitio web}
%~ \begin{framed}
	\begin{description}
		\item [Número:] \cn
		\item [Nombre:] Home del sitio web.
		\item [Categoría:] Web OMI.
		\item [Descripción:] La web OMI debe dispone de una página de entrada en la que se
      recoge: 
      \begin{itemize}
      \item Una descripción resumida del proyecto.
      \item Enlaces a las secciones principales.
      \item Un listado de noticias que será mantenida manualmente por el administrador
      \item Enlaces a las descargas de la última versión de los recursos.
      \end{itemize}
	\end {description}
%~ \end{framed}

\subsubsection{Sobre OMI}
%~ \begin{framed}
	\begin{description}
		\item [Número:] \cn
		\item [Nombre:] Sobre OMI.
		\item [Categoría:] Web OMI.
		\item [Descripción:] Deberá recoger información sobre el proyecto, el alcance del mismo, la autoría y los organismos implicados.
	\end {description}
%~ \end{framed}

\subsubsection{Contacto}
%~ \begin{framed}
	\begin{description}
		\item [Número:] \cn
		\item [Nombre:] Contacto.
		\item [Categoría:] Web OMI.
		\item [Descripción:] deberá listar diferentes medios de contacto. 
	\end {description}
%~ \end{framed}

\subsubsection{Índice de documentación}
%~ \begin{framed}
	\begin{description}
		\item [Número:] \cn
		\item [Nombre:] Índice de documentación.
		\item [Categoría:] Web OMI.
		\item [Descripción:]  Se ha de dispone de una página en la que se enlace de forma ordenada las distintas secciones de la 
      documentación, así como las distintas herramientas para navegar por esta.
	\end {description}
%~ \end{framed}

\subsubsection{Página de documentación}
%~ \begin{framed}
	\begin{description}
		\item [Número:] \cn
		\item [Nombre:] Índice de documentación.
		\item [Categoría:] Web OMI.
		\item [Descripción:]  Cada documento relativo a la presente memoria deberá estar disponible desde la web del proyecto.
	\end {description}
%~ \end{framed}

\subsubsection{Características}
%~ \begin{framed}
	\begin{description}
		\item [Número:] \cn
		\item [Nombre:] Características.
		\item [Categoría:] Web OMI.
		\item [Descripción:]  Se ha de disponer de una página en la que se liste las características del intérprete.
	\end {description}
%~ \end{framed}

\subsubsection{Navegador de gramática}
%~ \begin{framed}
	\begin{description}
		\item [Número:] \cn
		\item [Nombre:] Gramática.
		\item [Categoría:] Web OMI.
		\item [Descripción:]  Se ha de disponer de una página que sirva para navegar por las reglas gramaticales que definen el intérprete.
	\end {description}
%~ \end{framed}

\subsubsection{Navegador de clases}
%~ \begin{framed}
	\begin{description}
		\item [Número:] \cn
		\item [Nombre:] Clases.
		\item [Categoría:] Web OMI.
		\item [Descripción:]  Se ha de disponer de una página que sirva para navegar por las clases de programación sobre las que se construye el 
      intérprete.
	\end {description}
%~ \end{framed}

\subsubsection{Navegador de archivos}
%~ \begin{framed}
	\begin{description}
		\item [Número:] \cn
		\item [Nombre:] Archivos .
		\item [Categoría:] Web OMI.
		\item [Descripción:]  Se ha de disponer de una página que sirva para navegar por los archivos de código fuente que sirven para construir el intérprete.
	\end {description}
%~ \end{framed}

\subsubsection{Wiki}
%~ \begin{framed}
	\begin{description}
		\item [Número:] \cn
		\item [Nombre:] Wiki .
		\item [Categoría:] Web OMI.
		\item [Descripción:]  Se ha de disponer de una wiki para el proyecto que recoja aspectos relativos al proyecto y las materias que este estudia.
	\end {description}
%~ \end{framed}




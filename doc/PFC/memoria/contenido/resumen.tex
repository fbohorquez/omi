% ------------------------------------------------------------------------------
% Este fichero es parte de la plantilla LaTeX para la realización de Proyectos
% Final de Grado, protegido bajo los términos de la licencia GFDL.
% Para más información, la licencia completa viene incluida en el
% fichero fdl-1.3.tex

% Copyright (C) 2012 SPI-FM. Universidad de Cádiz
% ------------------------------------------------------------------------------

\thispagestyle{empty}

\noindent \textbf{\begin{Large}Resumen\end{Large}} 
\newline
\newline
\noindent 
El proyecto OMI tiene como razón de ser servir como apoyo en el aprendizaje de los conceptos detrás de los sistemas 
intérpretes. Se vale de la teoría de autómatas y los lenguajes formales para construir una serie de herramientas que ayudan en el proceso de estudio de estos sistemas. 
El proyecto lo conforman una serie de herramientas y recursos enmarcados dentro del campo de los intérpretes y lenguajes formales. \newline

OMI es un sistema intérprete del lenguaje de programación con el mismo nombre.  El intérprete es de código abierto y modular. Define un lenguaje orientado a objetos, de tipado dinámico
y con muchas otras características presentes en los lenguajes de programación modernos. Su objetivo es ilustrar el proceso de desarrollo y el funcionamiento interno que presenta un sistema intérprete para un
lenguaje de programación. \newline

El proyecto OMI presenta una serie de herramientas webs que permiten hacer uso del intérprete de forma online desde el navegador, detallándose los proceso sintácticos y semánticos llevados a cabo.
Además permiten navegar por toda la documentación de desarrollo del proyecto.  El proyecto pone a disposición de los desarrolladores la biblioteca de clases y  funciones sobre las que se escribe. 
Esta puede ser utilizada para la construcción de módulos o ser incluida en otros proyectos en los que se desee hacer uso del intérprete.  \newline

OMI es un proyecto que pretende acercar el mundo de los intérpretes y los lenguajes formales, mediante el uso de herramientas gráficas e ilustrativas, y 
facilitando una documentación desarrollada y precisa. El proyecto consiste en el diseño de un lenguaje de programación y el desarrollo del intérprete asociado.
El sistema software puede ser construido y ejecutado para que recibir código fuente y producir una salida correspondiente al proceso de 
interpretación, así otras herramientas, como las ofrecidas por el proyecto OMI, podrán usar esta capacidad para ilustrar el funcionamiento interno del interprete e 
interactuar con él. \newline

\noindent {\bf Palabras clave:} Intérprete, lenguaje formal, autómata, didáctico, ilustrativo, gramática, analizador léxico, analizador sintáctico, ejecución semántica, biblioteca de desarrollo .

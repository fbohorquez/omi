%fich.tex
%%%%%%%%%%%%%%%%%%%%%%%%%%%%%%%%%%%%%%%%%%%%%%%%%%%%%%%%%%%%%%%%%%%%%%%%
% Fco. Javier Bóhorquez Ogalla
% Descripción del documento:
	
%%%%%%%%%%%%%%%%%%%%%%%%%%%%%%%%%%%%%%%%%%%%%%%%%%%%%%%%%%%%%%%%%%%%%%%%


%%%%%%%%%%%%%%%%%%%%%%%%%%%%%%%%%%%%%%%%%%%%%%%%%%%%%%%%%%%%%%%%%%%%%%%%
%Clase de documento
\documentclass[12pt, spanish]{article}
%%%%%%%%%%%%%%%%%%%%%%%%%%%%%%%%%%%%%%%%%%%%%%%%%%%%%%%%%%%%%%%%%%%%%%%%


%%%%%%%%%%%%%%%%%%%%%%%%%%%%%%%%%%%%%%%%%%%%%%%%%%%%%%%%%%%%%%%%%%%%%%%%	
%Paquetes de lenguaje:
%%%%%%%%%%%%%%%%%%%%%%%%%%%%%%%%%%%%%%%%%%%%%%%%%%%%%%%%%%%%%%%%%%%%%%%%
\usepackage[utf8]{inputenc}
\usepackage[spanish]{babel}
%\usepackage[spanish, activeacute] {babel}	
%\usepackage[spanish]{babel} 				
%\usepackage[latin1]{inputenc}
%%%%%%%%%%%%%%%%%%%%%%%%%%%%%%%%%%%%%%%%%%%%%%%%%%%%%%%%%%%%%%%%%%%%%%%%


%%%%%%%%%%%%%%%%%%%%%%%%%%%%%%%%%%%%%%%%%%%%%%%%%%%%%%%%%%%%%%%%%%%%%%%%
%Paquetes para encabezados y pie de páginas
%%%%%%%%%%%%%%%%%%%%%%%%%%%%%%%%%%%%%%%%%%%%%%%%%%%%%%%%%%%%%%%%%%%%%%%%
\usepackage{fancyhdr}
%%%%%%%%%
% \pagestyle{fancy} %Si se cambia el estilo de página, antes de empezar 
	%el documento. Esto hace que el emcabezado y pie de página se 
	%quede separado del texto por una línea
%%%%%%%%%	
% \fancyhead{} % Límpia el texto que se está usando como encabezado
%%%%%%%%%
% \fancyhead[OPCIONES]{Encabezado}
	% OPCIONES:
		% L texto a la izquierda
		% C texto centrado
		% R texto a la derecha
		% E página par
		% O pagina impar
	%Ejemplo: 
		% fancyhead [LE] {"Doc. Latex"} 
			% Establece como emcabezado de las páginas pares "Doc Latex"
			% con alineacion a la izquierda
%%%%%%%%%			
% \fancyfoot[OPCIONES]{Píe de pag.}
	% OPCIONES:
		% L C R E O
	%Ejemplo
		% \fancyfoot[LE,RO]{\thepage} 
			% Establece como píe de pág. el número de página. con 
			% alineación a la izq en paginas pares y a la derecha en
			% las impares
%%%%%%%%%			
% \renewcommand{\headrulewidth}{0.4pt}
% \renewcommand{\footrulewidth}{0.4pt}
	% Fija el grosor de la la linea que separa el emcabezado y pie de pág
%%%%%%%%%
% Otros comandos:
%\lhead{}
%\chead{}
%\rhead{}
%\lfoot{}
%\cfoot{}
%%%%%%%%%%%%%%%%%%%%%%%%%%%%%%%%%%%%%%%%%%%%%%%%%%%%%%%%%%%%%%%%%%%%%%%%


%%%%%%%%%%%%%%%%%%%%%%%%%%%%%%%%%%%%%%%%%%%%%%%%%%%%%%%%%%%%%%%%%%%%%%%%
% Paquetes para tamaños y distancias
%%%%%%%%%%%%%%%%%%%%%%%%%%%%%%%%%%%%%%%%%%%%%%%%%%%%%%%%%%%%%%%%%%%%%%%%
\usepackage{anysize} 
%%%%%%%%%%%
% \marginsize{3cm}{2cm}{2cm}{2cm}
	% Controla los márgenes {izquierda}{derecha}{arriba}{abajo}. 
%%%%%%%%%%%%%%%%%%%%%%%%%%%%%%%%%%%%%%%%%%%%%%%%%%%%%%%%%%%%%%%%%%%%%%%%


%%%%%%%%%%%%%%%%%%%%%%%%%%%%%%%%%%%%%%%%%%%%%%%%%%%%%%%%%%%%%%%%%%%%%%%%
%Paquetes de carácteres especiales:
%%%%%%%%%%%%%%%%%%%%%%%%%%%%%%%%%%%%%%%%%%%%%%%%%%%%%%%%%%%%%%%%%%%%%%%%
\usepackage{dsfont}	
%Para representar conjuntos matematicos comunes: Z, N, R...
	% mathds{R}, mathds{N},...
%%%%%%%%%%%%%%%%%%%%%%%%%%%%%%%%%%%%%%%%%%%%%%%%%%%%%%%%%%%%%%%%%%%%%%%%

%%%%%%%%%%%%%%%%%%%%%%%%%%%%%%%%%%%%%%%%%%%%%%%%%%%%%%%%%%%%%%%%%%%%%%%%
%Paquetes para insertar gráficos
%%%%%%%%%%%%%%%%%%%%%%%%%%%%%%%%%%%%%%%%%%%%%%%%%%%%%%%%%%%%%%%%%%%%%%%%
\usepackage[dvips]{graphicx}
\DeclareGraphicsExtensions{.pdf,.png,.jpg} %solo para PDFLaTeX
%%%%%%%%%%%%%%%%%%%%%%%%%%%%%%%%%%%%%%%%%%%%%%%%%%%%%%%%%%%%%%%%%%%%%%%%

%%%%%%%%%%%%%%%%%%%%%%%%%%%%%%%%%%%%%%%%%%%%%%%%%%%%%%%%%%%%%%%%%%%%%%%%
%Paquetes de color:
%%%%%%%%%%%%%%%%%%%%%%%%%%%%%%%%%%%%%%%%%%%%%%%%%%%%%%%%%%%%%%%%%%%%%%%%
\usepackage{color}
%%%%%%%%%%%%%%%%%%%%%%%%%%%%%%%%%%%%%%%%%%%%%%%%%%%%%%%%%%%%%%%%%%%%%%%%
%Definición de colores:
\definecolor{gray97}{gray}{.97}
\definecolor{gray75}{gray}{.75}
\definecolor{gray45}{gray}{.45}
%%%%%%%%%%%%%%%%%%%%%%%%%%%%%%%%%%%%%%%%%%%%%%%%%%%%%%%%%%%%%%%%%%%%%%%%


%%%%%%%%%%%%%%%%%%%%%%%%%%%%%%%%%%%%%%%%%%%%%%%%%%%%%%%%%%%%%%%%%%%%%%%%
%Paquete de listado de código:
%%%%%%%%%%%%%%%%%%%%%%%%%%%%%%%%%%%%%%%%%%%%%%%%%%%%%%%%%%%%%%%%%%%%%%%%
\usepackage{listings}
%%%%%%%%%%%%%%%%%%%%%%%%%%%%%%%%%%%%%%%%%%%%%%%%%%%%%%%%%%%%%%%%%%%%%%%%
%Configuración del listado:
\lstset { 
	frame=Ltb,
     framerule=0pt,
     aboveskip=0.2cm,
     framextopmargin=0.3pt,
     framexbottommargin=0.2pt,
     framexleftmargin=0.3cm,
     framesep=0pt,
     rulesep=.1pt,
     tabsize=2,
     backgroundcolor=\color{gray97},
     rulesepcolor=\color{black},
     %
     stringstyle=\ttfamily,
     showstringspaces = false,
     basicstyle=\scriptsize\ttfamily,
     commentstyle=\color{gray45},
     keywordstyle=\bfseries,
   	%
     numbers=left,
     numbersep=1pt,
     numberstyle=\tiny,
     numberfirstline = false,
     breaklines=true,
}
%%%%%%%%%%%%%%%%%%%%%%%%%%%%%%%%%%%%%%%%%%%%%%%%%%%%%%%%%%%%%%%%%%%%%%%%


%%%%%%%%%%%%%%%%%%%%%%%%%%%%%%%%%%%%%%%%%%%%%%%%%%%%%%%%%%%%%%%%%%%%%%%%
%Indexado. Índice alfabeticos
\usepackage{makeidx}
\makeindex %para habilitar indices
%%%%%%%%%%%%%%%%%%%%%%%%%%%%%%%%%%%%%%%%%%%%%%%%%%%%%%%%%%%%%%%%%%%%%%%%
%Forma de uso:
% \index{clave} %Para crear un indice de materia
% Supongase que queremos meter en el indice alfabetico referencias a 
% las paginas donde se hace referencia a la clave Producto escalar
% en tal caso en cada una de las páginas pondremos \index{Producto escalar}
% \printindex %Para imprimir el índice alfabetico

% Es necesario una doble compilación del documento. La primera genera un 
% fichero.idx el cual tendremos que procesar con la aplicación makeindex
%tras procesarlo se creará un fichero.ind que contendra el código Latex 
% que se inserta en el documento original. 
% Tras la segunda compilación del documento original se sustituira el
% comando \printindex por el contenido del fichero.ind 

%Podemos cambiar el formato de la clave:
%Ejemplo     				&    	Entrada 		&	Comentario
%\index{hola}				&         hola, 1   	&	Entrada simple
%\index{hola!Pedro}			&		 Pedro, 3 	&	Subentrada bajo ‘hola’
%\index{Juan@\textsl{Juan}}  	&		Juan, 2    	&	Entrada con diseño                        
%\index{Pepa@\textbf{Pepa}}	&		Pepa, 7    	&	Igual que antes
%\index{Loli|textbf}		&		Loli, 3    	&	No de página con diseño                     
%\index{Soraya|textit}		&		Soraya, 5  	&	Igual que antes
%%%%%%%%%%%%%%%%%%%%%%%%%%%%%%%%%%%%%%%%%%%%%%%%%%%%%%%%%%%%%%%%%%%%%%%%


%%%%%%%%%%%%%%%%%%%%%%%%%%%%%%%%%%%%%%%%%%%%%%%%%%%%%%%%%%%%%%%%%%%%%%%%


%%%%%%%%%%%%%%%%%%%%%%%%%%%%%%%%%%%%%%%%%%%%%%%%%%%%%%%%%%%%%%%%%%%%%%%%
%Estilo de página
%%%%%%%%%%%%%%%%%%%%%%%%%%%%%%%%%%%%%%%%%%%%%%%%%%%%%%%%%%%%%%%%%%%%%%%%
%\textwidth 6.75in								  %ancho de texto
%\oddsidemargin -0.2in							%margen izquierdo 
\parskip 0.2in									%espacio parrafos
%%%%%%%%%%%%%%%%%%%%%%%%%%%%%%%%%%%%%%%%%%%%%%%%%%%%%%%%%%%%%%%%%%%%%%%%

\usepackage{titlesec}

\setcounter{secnumdepth}{4}

\titleformat{\paragraph}
{\normalfont\normalsize\bfseries}{\theparagraph}{1em}{}
\titlespacing*{\paragraph}
{0pt}{3.25ex plus 1ex minus .2ex}{1.5ex plus .2ex}


%%%%%%%%%%%%%%%%%%%%%%%%%%%%%%%%%%%%%%%%%%%%%%%%%%%%%%%%%%%%%%%%%%%%%%%%	
%datos del documento
\author{Fco. Javier Bohórquez Ogalla}						%autor
\date{}														%fecha

\title{Ilos}
%%%%%%%%%%%%%%%%%%%%%%%%%%%%%%%%%%%%%%%%%%%%%%%%%%%%%%%%%%%%%%%%%%%%%%%%

%%%%%%%%%%%%%%%%%%%%%%%%%%%%%%%%%%%%%%%%%%%%%%%%%%%%%%%%%%%%%%%%%%%%%%%%
%Documento:
%%%%%%%%%%%%%%%%%%%%%%%%%%%%%%%%%%%%%%%%%%%%%%%%%%%%%%%%%%%%%%%%%%%%%%%%
\begin{document}
\maketitle
\pagebreak
\tableofcontents
\pagebreak
\section {Título}
Rules. Lenguaje para el desarrollo software dentro del ámbito de los juegos de mesa.
Rules. Lenguaje para la programación de juegos de mesa.
Rules. Lenguaje para el desarrollo de juegos de mesa.
\section {Documentos básicos}
\begin{itemize}
	\item Indice General
	\item Memoria
	\item Anexos
	\item Planos
	\item Pliego de condiciones
	\item Estado de mediciones
	\item Presupuesto
	\item Estudio con entidad propia
\end{itemize}


\section{Volúmenes}
Los documentos pueden estar agrupados en distintos volumenes o en uno solo.

Por cada volumen
\begin {itemize}
	\item Número de volumen
	\item Título del proyecto
	\item Tipo de documento básico unitario que comprende.
	\item Organismo o cliente
	\item Identificación y datos profecionales del autor
	\item Cuando corresponda, los de la persona jurídica que ha recibido 
	el encargo de su elaboración.
\end {itemize}

\subsection{Páginas}
\begin{itemize}
\item Número de página o de plano.
\item Título del Proyecto o Número o código de identificación del Proyecto.
\item Título del documento básico a que pertenece.
\item Número o código de identificación del documento.
\item Número de edición y, en su caso, fecha de aprobación.
\end{itemize}

\subsection {Capítulos y apartados}
Segun la norma UNE 50132. (1.2.3)
\subsection {Redacción}
\begin{itemize}
	\item Lenguaje claro, preciso y libre de vagedadez
	\item Terminología coherente
	\item El tiempo futuro indicará requisitos obligatorios. 
	\item El tiempo condicional o subjuntivo indicarán sugerencias o propuestas
\end{itemize}

\section{Desarrollo de documentos}
\subsection{Índice general}
Contendrá todos y cada uno de los índices de los diferentes documentos básicos.
\subsection{Memoria}
\begin{itemize}
	\item Hoja de identificación (título, datos autor, datos de cliente, fecha y firma)
	\item Indice de la memoria (indice de cada documento, capítulo o apartado)
	\item Objeto
	\item Alcance
	\item Antecedentes
	\item Normas y referencias
	\item Definiciones y abreviaturas
	\item Requisitos de diseño
	\item Análisis de soluciones
	\item Resultados finales
	\item Planificación
	\item Orden de prioridad entre los documentos básicos
\end{itemize}

\subsubsection {Objeto}
Comentar el objeto del proyecto desde el punto de vista de la necesidad que pretende cubrir.
Centrarse en la tarea de digitalizar juegos y la necesidad de tener recursos que faciliten esta labor.
\subsubsection {Alcance}
\begin{itemize}
	\item Centrarse en la diversidad de juegos de mesas. 
	\item Rules se centra en la definición de reglas y el modelado lógico de los componentes. 
	\item Rules ofrece mecanismos para tratar aspectos comunes en muchos de ellos. 
	\item Existen particularidades de muchos juegos que rules no contempla (Física...).
\end{itemize}

\subsubsection {Antecedentes}
\begin{itemize}
	\item Lenguajes formales
	\item Compiladores y traductores
	\item Algoritmia
	\item Estrucutras de datos
	\item Base de datos
	\item Aspectos de los lenguajes de programación
	\item ORMs
\end{itemize}

\subsubsection {Normas y referencias}
\begin {itemize}
	\item Disposiciones legales y normas aplicadas
	\begin {itemize}
		\item Licencias a las que está sujeta cada software utilizada
		\item Licencias sobre las que se desarrolla el proyecto
	\end {itemize}
	\item Bibliografía
	\begin {itemize}
		\item libro. Compiladores y procesadores del lenguaje
		\item documento. Manual GNU Bison
		\item documento. Manual GNU Flex
		\item artículo. Wikipedia. Analizador léxico
		\item artículo. Wikipedia. Analizador sintáctico
	\end {itemize}
	\item Programas de cálculo ???
	\item Plan de gestión de calidad
	\begin{itemize}
		\item Pruebas unitarias
		\item Pruebas de rendimiento 
	\end{itemize}
	\item Otras referencias
\end {itemize}

\subsubsection{Definiciones y abreviaturas }
\subsubsection{Requisitos de diseño}
\begin{itemize}
	\item Requisitos funcionales: Casos de usos
	\item Requisitos no funcionales
	\begin{itemize}
		\item Sistema de base de datos portable
		\item Rendimiento
		\item Minimizar dependencias prácticas con otros sistemas software
		\item Software libre
		\item Interacción con base de datos relacional
	\end{itemize}
\end{itemize}

\subsubsection{Analisis de soluciones}
\begin{itemize}
	\item Análisis de los casos de uso
	\item Decisiones de diseño
\end{itemize}
\subsubsection{Resultados finales}
\begin{itemize}
	\item Tokens y analizador léxico
	\item Gramática y analizador sintáctico
	\item Tabla de símbolos
	\item Nodos semánticos ejecutables
	\begin{itemize}
		\item Generales
		\begin{itemize}
			\item Nodo ejecutable
			\item Nodo imprimible
			\item Nodo expresión (definida y no definida)
			\item Nodo lógico
			\item Nodo aritmetico
			\item Nodo textual
		\end{itemize}
		\item Tipos simples
		\begin{itemize}
			\item Booleano
			\item Numerico
			\item Carácter
			\item Valor nulo
		\end{itemize}
		\item Tipos compuestos
		\begin{itemize}
			\item Arrays
			\item Cadenas
			\item Expresiones regulares
			\item Objetos (prototipado)
			\item Componente
		\end{itemize}
		\item Definiciones
		\begin{itemize}
			\item Ámbito de variables
			\item Funciones (parámetros por defecto y por referencia, funciones anónimas, referencas a funciones)
			\item Clases (herencia, new , this)
			\item Componentes
		\end{itemize}
		\item Sentencias de control de flujo
		\begin {itemize}
			\item if
			\item if else
			\item if elif ... else
			\item case
			\item for
			\item while
			\item do while
			\item iloop (variable de control iterativo)
			\item foreach
			\item goto (labels)
			\item include
			\item break
			\item continue <======
			\item return
			\item exit
			\item Llamada a función
		\end {itemize}
		\item Generadores de expresiones
		\item Sentencias operacionales
		\begin{itemize}
			\item Tipos
			\begin{itemize}
				\item typeof
				\item bool
				\item int
				\item float
				\item size
			\end{itemize} 
			\item Datos 
			\begin{itemize}
				\item Asignación
				\item Asignación por referencia
				\item Ternario (simplificado, con valor por defecto nulo)
				\item No nulo
				\item rand <====
			\end{itemize}
			\item Entrada/Salida
			\begin{itemize}
				\item echo (salida estandar)
				\item input (entrada estándar, prompt)
				\item inputline (entrada estándar, prompt)
			\end{itemize} 
			\item Operaciones lógicas
			\begin{itemize}
				\item OR lógico
				\item OR bit a bit  <======
				\item OR bit a bit y asignación  <======
				\item AND lógico
				\item AND bit a bit <======
				\item AND bit a bit y asignación <======
				\item NOT Lógico
				\item Igualdad
				\item Desigualdad
				\item Menor que
				\item Menor o igual que
				\item Idéntico <========
				\item No identico <=======
			\end{itemize} 
			\item Operaciones aritméticas
			\begin{itemize}
				\item Suma 
				\item Resta 
				\item Producto 
				\item División 
				\item Potencia 
				\item Módulo
				\item Incremento y asignación
				\item Asignación e incremento
				\item Decremento y asignación
				\item Asignación y decremento
				\item Suma y asignación 
				\item Resta y asignación 
				\item Producto y asignación 
				\item División y asignación
				\item Potencia y asignación
				\item Módulo y asignación
			\end{itemize} 
			\item Cadenas
			\begin{itemize}
				\item Concatenación
				\item Concatenación y asignación
				\item sprintf
				\item find
				\item replace
				\item upper
				\item lower
			\end{itemize} 
			\item Arrays 
			\begin{itemize}
				\item implode
				\item explode
				\item empty
			\end{itemize}
			\item Expresiones regulares
			\begin{itemize}
				\item match
				\item search
				\item replace
			\end{itemize}
			\item Base de datos <====
			\begin{itemize}
				\item connect
				\item select
				\item insert
				\item update
				\item delete
			\end{itemize}
			\item Fechas  <====
			\begin {itemize}
				\item date
				\item time
				\item sleep
			\end {itemize}
			\item Ficheros <====
			\begin {itemize}
				\item fopen
			\end {itemize}
			\item Llamadas al sistema
			\begin{itemize}
				\item Ejecución de comando (exec)
				\item Interpretación de cadena (eval)
			\end{itemize}
			\item Procesos
			\begin{itemize}
				\item fork
			\end{itemize}
			\item Componentes
			\begin{itemize}
				\item Acceso
				\item Búsqueda
				\item Borrado
				\item Asignación de atributos
			\end{itemize}
		\end{itemize} 
		\item Control de errores <====
	\end{itemize}
	\item Sistema de eventos <====
	\item Sistema de juego <===
	\begin {itemize}
		\item Definición de juego 
		\begin{itemize}
			\item Bloque de juego
			\item Bloque de juego con cambio de turno
			\item Bloque de inicialización de juego
			\item Bloque de finalización de juego
			\item Condiciones de finalización
			\item Acotación de jugadores
		\end{itemize}
		\item Jugadores (jugadores totales, jugador activo, iteraciones con cambios de jugador)
		\item Turnos (cambio de turno, penalización de turnos, cambios con regreso)
		\item Posicionamiento lógico (zona de juego, mesa)
		\item Flujos de movimiento
		\item Control de flujo de juego (finalización de ronda o de juego)
		\item Bloque de deciciones y opciones (decide, select)
		\item Dados
		\item Mazos (barajar, robar, descarte, mirar componentes, buscar componentes...)
		\item Reservas (barajar, robar, devolver, mirar componentes, buscar componente...)
	\end {itemize}
	\item Gestión de memoria
\end{itemize}
\subsubsection{Planificación}
Se definirán las diferentes etapas, metas o hitos a alcanzar, plazos de entrega y cronogramas o gráficos de programación correspondientes
Diagrama de gant, planner. Rango de un año poco más.

\subsubsection{Orden de prioridad de documentos básicos}
Si no se indica será el siguiente:
\begin{itemize}
	\item Planos
	\item Pliego de Condiciones
	\item Presupuesto
	\item Memoria
\end{itemize} 
\subsection{Anexos}
Se compone de los siguientes documentos
\begin {itemize}
	\item Documentación de partida (doc. para generar los requisitos, reglamentos de juegos)
	\item Cálculos. (justificación de soluciones: gestión de memoria, tamaño de tipos, algoritmos)
	\item Anexos de aplicación en el ámbito del Proyecto. (licencias, seguridad)
	\item Otros documentos: (Juegos desarrollados)
\end {itemize}

\subsection{Planos}
Índice que hará referencia a cada uno de ellos, indicando su ubicación, con el fin de facilitar su utilización.
\begin{itemize}
	\item Diagramas de casos de uso
	\item Diagrama conceptual de clases
	\item Diagrama de clases
	\item ... (ver libro UML)
\end{itemize}

\subsection{Pliego de condiciones}
Condiciones que deben darse para que el objetivo del proyecto pueda materializarse
\subsection {Estado de mediciones}
Desglose del proyecto en unidades de obras
\subsection{Presupuesto}
\begin{itemize}
\item Cuadro de precios unitarios de materiales, mano de obra y elementos auxiliares.
\item Cuadro de precios unitarios de materiales, mano de obra y elementos auxiliares por unidades de obras.
\item Presupuesto: valoración económica global, desglosada y ordenada según el Estado de mediciones.
\end{itemize}
Aspectos a considerar
\begin{itemize}
	\item Gastos generales y beneficio industrial
	\item Impuestos, tasas y otras contribuciones
	\item Seguros
	\item Costes de certificación y visado
	\item Permisos y licencias
	\item Cualquier otro concepto que influya en el coste final de materialización del objeto del Proyecto.
\end{itemize}

\subsection {Estudios con entidad propia}
Documentos requeridos por exigencias legales.

\end{document}
%%%%%%%%%%%%%%%%%%%%%%%%%%%%%%%%%%%%%%%%%%%%%%%%%%%%%%%%%%%%%%%%%%%%%%%%
%Fco. Javier Bohórquez Ogalla

% ------------------------------------------------------------------------------
% Este fichero es parte de la plantilla LaTeX para la realización de Proyectos
% Final de Grado, protegido bajo los términos de la licencia GFDL.
% Para más información, la licencia completa viene incluida en el
% fichero fdl-1.3.tex

% Copyright (C) 2012 SPI-FM. Universidad de Cádiz
% ------------------------------------------------------------------------------

En esta sección se detallan las lecciones aprendidas tras el desarrollo del proyecto OMI y
se identifican las posibles mejoras futuras sobre el desarrollo del software.
% ======================================================================
\section{Objetivos alcanzados}
Como se introdujo se ha construido un intérprete completo de un lenguaje de programación de 
alto nivel, capaz de describir mediante estructuras de datos el proceso de interpretación llevado a cabo. También se 
ha construido un cliente web capaz de interpretar estos datos y representarlos gráficamente y de forma interactiva.

Se ha recorrido todo el proceso de desarrollo y construcción del proyecto. La documentación generada ha sido dispuesta 
en una plataforma web que permite su acceso y navegación de una forma fácil y cómoda. 

\section{Lecciones aprendidas}
\begin{itemize}
\item Se ha profundizado en los estudios de los intérpretes de lenguajes de formales y los conceptos que los hacen posibles tales como 
los autómatas, el léxico, la sintaxis, las gramáticas, los autómatas, los árboles...
\item Se han estudiado y asimilado características avanzadas de los lenguajes de programación y cómo estas se implementan.
\item Se ha profundizado en otros lenguajes de programación para construir una visión amplia de características y paradigmas (PHP, Python, JavaScript, Java, Haskell, Ruby...). 
\item Se han afianzados nuevos conocimientos en el desarrollo C++ y algunas características avanzadas de este, como la inclusión de bibliotecas dinámicas, bibliotecas 
de líneas de comandos o las directivas del compilador.
\item Se ha estudiado y profundizado en herramientas de compilación automática y la paquetización de binarios precompilados.
\item Se ha profundizado en la instalación y configuración de servidores.
\item Se han estudiado estructuras de datos estándar utilizadas en comunicaciones como JSON.
\item Se ha profundizado en el proceso de desarrollo de un proyecto software y las metodologías aplicadas.
\item Se ha profundizado en el desarrollo de diagramas y el lenguaje UML.
\item Se han estudiado estándares ISO para asegurar la calidad de proyectos.
\item Se ha adquirido conocimientos más afianzados en los distintos tipos de pruebas aplicables al software.
\item Se ha profundizado en la planificación y desarrollo de tareas.
\item Se ha profundizado en el desarrollo de webs dinámicas escritas en PHP.
\item Se ha adquirido nuevos conocimientos en tecnologías de navegadores tales como HTML5 y JavaScript.
\item Se ha adquirido destreza para el trabajo en equipo y una comunicación óptima.
\end{itemize}

\section{Trabajo futuro}
\begin{itemize}
\item Añadir características pertenecientes a los paradigmas abordados como número de parámetros arbitrarios, acceso a variables no locales, herencia múltiple, traits, prototipos...
\item Añadir características de otros paradigmas no abordados como la programación dirigida por eventos, aspectos, lógica ...
\item Añadir características avanzadas al intérprete y su funcionamiento interno como la compilación justo a tiempo.
\item OMI es un lenguaje de ámbito general, pero puede ser especializado para cumplir un propósito más concreto como la explotación de servicios web o la creación de juegos. 
\item Desarrollar en mayor detalle una descripción del proceso de interpretación entrando en materias como el análisis léxico y sintáctico llevado a cabo. 
\item Integrar un DSL para la creación de gramáticas dentro del propio lenguaje.
\item Construir bibliotecas de funciones útiles que añadan recursos matemáticos, de gestión de bases de datos...
\end{itemize}

% ------------------------------------------------------------------------------
% Este fichero es parte de la plantilla LaTeX para la realización de Proyectos
% Final de Grado, protegido bajo los términos de la licencia GFDL.
% Para más información, la licencia completa viene incluida en el
% fichero fdl-1.3.tex

% Copyright (C) 2012 SPI-FM. Universidad de Cádiz
% ------------------------------------------------------------------------------

El proyecto OMI ofrece un conjunto de herramientas y recursos que ayudan en el aprendizaje 
de la teoría de intérpretes y los lenguajes formales. OMI en si mismo es un intérprete desarrollado 
con la finalidad de servir como caso práctico, aplicando los conceptos estudiados en 
estos campos. Es un intérprete modular, en el sentido de que su funcionalidad y características puede ser extendida mediante módulos.
OMI es código abierto y libre, por lo que puede ser utilizado, estudiado, modificado y/o redistribuido libremente.\\

El nombre OMI también hace referencia al lenguaje que es interpretado. Un lenguaje de programación de alto nivel, de tipado dinámico y de propósito general. Fundamentalmente 
imperativo orientado a objetos, pero que además presenta características propias de otros paradigmas como de la programación funcional. \\

El proyecto OMI pretende facilitar el estudio de los sistemas intérpretes mediante la documentación y la interactividad. Abarcando no solo el diseño y el desarrollo
de uno de estos sistemas, si no el proceso de interpretación en si. De esta forma se presentan herramientas que permiten hacer uso del intérprete a la vez que ilustran
su funcionamiento. 

\section{Motivación}
La teoría de autómatas y los lenguajes formales son la base para la construcción de sistemas intérpretes. Muchos de los conceptos estudiados 
en estos campos sirven para construir otras ramas de estudio como la teoría de intérpretes y compiladores. 

Normalmente, por limitaciones de tiempo, los cursos académicos relacionados con el estudio de estos campos están enfocados a los conceptos teóricos. 
Llevándose a cabo tan solo algunas prácticas que refuerzan el aprendizaje y ayudan a ver la aplicación real mediante casos sencillos y simplificados. 
Sería de gran ayuda en este proceso disponer de un caso práctico completo, que ilustre cómo se define y construye un lenguaje de programación actual, 
y que pueda ser consultado cuándo, dónde y por quién lo desee. OMI pretende ser una fuente de información que complemente el estudio práctico en estas áreas.\\

Aunque existen herramientas que dan soporte a la construcción de sistemas intérpretes, es difícil encontrar alguna que ayude a comprender cómo estos se construyen y funcionan. 
Los intérpretes modernos no están enfocados en ilustrar la tarea que llevan a cabo, tienen como propósito ejecutar programas de forma óptima y efectiva. Sería de utilidad disponer de una 
herramienta que documente los procesos sintácticos y semánticos llevados a cabo durante la interpretación de código fuente. OMI tiene como propósito generar información relativa al proceso 
de interpretación que muestre el funcionamiento del intérprete. 

La comunidad académica no solo precisa de un objeto de estudio, si no también de un objeto que ayude a formar la comunidad. El proyecto OMI pretende ser una base para crear una comunidad 
centrada en el estudio de los lenguajes formales, en las que todos puedan contribuir a la vez de beneficiarse.

\section{Alcance}

El proyecto OMI abarca una serie de herramientas y recursos que ayudan a comprender 
cómo se construye y funciona un sistema intérprete.

\subsection{Intérprete}
El intérprete OMI es la parte fundamental del proyecto, las demás herramientas y recursos se valen o se construyen a partir de este. 
Es un sistema software capaz de analizar y ejecutar otros programas escritos en un lenguaje específico con el mismo nombre.

El intérprete puede procesar código fuente recibido mediante la entrada estándar. También es posible ejecutarlo como una terminal interactiva, mostrando un prompt
e interpretando el código introducido. Por otro lado podrá ser configurado para obtener el código fuente mediante un puerto TCP, funcionando así como un software 
servidor. El software puede recibir una serie de parámetros que serán pasados como argumentos al programa.  Además puede ser ejecutado de forma que produzca una salida en formato JSON relativa a acciones llevadas a cabo durante el proceso.

El intérprete puede ser extendido mediante módulos lo que permite ampliar las posibilidades del lenguaje subyacente. Por otro lado el software 
puede ser configurado desde el momento en el que es construido, de forma que puede prescindir y/o alterar algunas de sus características, adaptándose mejor 
al uso concreto que se le fuera a dar. 

\subsection{Lenguaje} 
OMI es un lenguaje multiparadigma de alto nivel y de propósito general, que presenta un tipado dinámico. Su sintaxis pretende 
ser sencilla y similar a los lenguajes de programación modernos. 
Contempla tipos de datos simples y compuestos, así como un conjunto de operaciones sobre estos. Los tipos de datos que pueden ser 
manipulados por el intérprete OMI comprenden:
\begin{itemize}
  \item Lógicos
  \item Aritméticos
  \item Cadena de caracteres
  \item Array de elementos
  \item Expresiones regulares
  \item Funciones
  \item Referencias
  \item Objetos
\end{itemize}

El intérprete OMI soporta variedad de operadores que trabajan sobre los tipos de datos indicados. Estos son aquellos que se pueden encontrar de forma
nativa en cualquier lenguaje de programación actual, a excepción de los operadores booleanos bit a bit. Esto incluye operadores tales como 
la suma, la potencia, la negación, la concatenación, el ternario, fusión de nulos... Se dispone además de una serie de operadores para la entrada/salida de datos. 

También permite al usuario asignar valores a símbolos variables, que pueden ser utilizados posteriormente en otras expresiones y construcciones del lenguaje. Las 
variables pueden ser locales o globales.

El sistema intérprete es capaz de procesar estructuras de control tales como if, while, switch... Además presenta otras sentencias que operan sobre el flujo de ejecución, 
como por ejemplo la inclusión de ficheros, la rotura de bloques, la continuación de bucle o la parada de la ejecución. También permite la definición de etiquetas y el salto a las mismas.\\

En OMI es posible definir y realizar llamadas a funciones. Estas pueden hacer uso de parámetros por defecto, y estos pueden 
ser pasados por referencia o valor.

OMI es un lenguaje orientado a objetos. Permite la definición de clases y la instanciación de estas en objetos. Contempla algunas características 
propias de este paradigma:
\begin{itemize}
\item Definición de métodos y atributos
\item Método constructor
\item Visibilidad pública y privada
\item Herencia simple
\item Métodos y atributos estáticos
\item Enlace estático en tiempo de ejecución
\item Métodos mágicos
\item Duck typing
\item Tipos de datos básicos como clase de objetos
\end{itemize}

OMI además presenta características propias de la programación funcional:
\begin{itemize}
\item Funciones de primer orden
\item Clausura de funciones
\item Decoradores
\item Pliegues
\item Aplicación parcial
\item Funciones anónimas
\item Listas por compresión
\end{itemize}

OMI es un lenguaje con mecanismos que favorecen la reflexión y permiten llevar a cabo introspección de tipos. 

El lenguaje ofrece una serie de recursos en forma de funciones destinados a la gestión de
ficheros, procesos y fechas.
  
\subsection{Módulos}
OMI puede extender su funcionalidad y características mediante el uso de módulos. El proyecto OMI incluye 
el desarrollo de uno de estos módulos y la documentación de este proceso.

El módulo gettext añade funciones para la internacionalización de programas. Implementa las funciones de la biblioteca de GNU
con el mismo nombre.

\subsection{Biblioteca de desarrollo}
La biblioteca de desarrollo OMI incluye todos los recursos de programación necesarios para construir el intérprete. 
Está escrita en C++, y se puede incluir en cualquier proyecto software escrito en este lenguaje. 
Esta biblioteca puede ser instalada de forma independiente al intérprete, y puede ser usada para el desarrollo de 
módulos o cualquier otro sistema software. Permite construir nodos semánticos a partir de los ya definidos e 
interpretar código fuente escrito en OMI.

\subsection{Sitio web}
El proyecto OMI incluye un sitio web que sirve como presentación del mismo, además de como medio de acceso a la documentación 
y el software desarrollado. Todas las páginas web pertenecientes al sitio contienen información relativa al proyecto y a las áreas que este
ocupa. 
\begin{description}
\item [Página de inicio:] Describe e introduce brevemente el proyecto. Contiene enlaces a las demás secciones del sitio web. Además presenta un
listado de noticias y enlaces de descargas a la última versión del intérprete.
\item [Índice de documentación:] Página que representa un índice de los documentos que conforman el proyecto.
\item [Documentos:] Páginas relativas a la documentación del proyecto en si.
\item [Navegador de clases:] Páginas relativas a la documentación de las clases incluidas en la biblioteca. 
\item [Navegador de ficheros:] Páginas relativas a la documentación de los ficheros que conforman 
el código fuente de la biblioteca y el intérprete.
\item [Navegador de gramática:] Páginas relativas a la documentación gráfica de la gramática del lenguaje.
\item [Descargas:] Página que enlaza la descarga de las distintas versiones del software que conforma el proyecto, disponibles en varios formatos de instalación.
\item [Sobre OMI:] Página con información relativa a la motivación y circunstancias en las que se ha dado el proyecto. Además da detalles sobre los autores y los
organismos implicados en el desarrollo del mismo.
\item [Contacto:] Página con información de contacto.
\end{description}

El sitio web también sirve como enlace a herramientas que permiten hacer uso del intérprete de forma online.

%~ El sitio OMI presenta un di seño adaptativo, por lo que su apariencia se adapta al dispositivo que accede al mismo. 

\subsection{RunTree} 
Herramienta online que permite escribir código OMI e interpretarlo. La herramienta describe el árbol sintáctico resultado del análisis del código fuente, y lleva a cabo la ejecución
semántica del mismo paso a paso. Además muestra información de todo el proceso, incluyendo su estado interno y la entrada/salida de datos. 
Esta aplicación representa un cliente del intérprete OMI cuando este es ejecutado como un servidor. Se comunica de forma distribuida con el intérprete y procesa los datos devueltos por el mismo
para mostrarlos al usuario de una forma gráfica y ordenada.   

\section{Organización del documento}
La documentación recogida en esta memoria presenta el siguiente contenido y organización:

\begin {description}
\item [Planificación:] Se describe la gestión del proyecto. Aspectos tales como la metodología, organización, 
los costes, la planificación, los riesgos y el aseguramiento de la calidad. 
\item [Requisitos del sistema:] Se describe las necesidades y la situación que han originado el desarrollo del proyecto. Además 
se detallan los objetivos y los requisitos que debe cumplir el sistema. 
\item [Análisis del sistema:] Se lleva a cabo el modelado  relativo al análisis del sistema. Haciendo uso de diagramas UML se 
describe el modelo conceptual de datos, el modelo de casos de uso, el modelo de comportamiento y la interfaz de usuario. 
\item [Diseño del sistema:] Aborda la arquitectura lógica y física del sistema, la descripción de la gramática, el diseño de componentes y la
interfaz de usuario.
\item [Construcción del sistema:] En esta sección se describe el entorno de construcción y la organización del código.
\item [Pruebas del sistema:] Se presenta el plan de pruebas seguido, especificando la estrategia seguida, el entorno de pruebas y los distintos
niveles de pruebas realizadas.
\item [Manual de implantación:] Describe cómo se ha de llevar a cabo la implantación de los distintos sistemas software que 
conforman el proyecto para la puesta en producción. 
\item [Manual de usuario:] Representa una guía de referencia para el correcto uso del software.
\item [Conclusiones:] Se detallan los objetivos alcanzados y las lecciones aprendidas en el desarrollo del proyecto.
\item [Bibliografía:] Lista y recapitula la bibliografía empleada en el desarrollo del proyecto.
\item [Información sobre licencia:] Se presenta y detalla información sobre la licencia de uso del software y la documentación.
\end{description}

Por otro lado el conjunto de software desarrollado estará disponible desde varios canales y formatos:
\begin{itemize}
\item CD 
\item Forja de software
\item Apartado de descargas en la web
\end{itemize}
% ======================================================================

   

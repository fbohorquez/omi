\subsection{Asignaciones}
\subsubsection{Asignación}
%\begin{framed}
	\begin{description}
		\item [Número:] \cn
		\item [Nombre:] Asignación.
		\item [Categoría:] Asignaciones.
		\item [Descripción:] Se hace necesario la gestión de los símbolos variables creados durante la ejecución, lo que implica
		la asignación de valores a las variables que serán definidas y utilizadas por el contenido fuente dado por el usuario.
		El valor que es asignado a una variable puede ser cualquier tipo de dato contemplado, incluso funciones o objetos. El valor asignado
		puede ser determinado a partir de cualquier expresión que tenga un valor asociado después de su ejecución.  La operación de asignación debe ser en si misma una expresión que toma como
		valor tras su ejecución el valor asignado.
	\end {description}
%\end{framed}

\subsubsection{Asignación de referencia}
%\begin{framed}
	\begin{description}
		\item [Número:] \cn
		\item [Nombre:] Asignación de referencia.
		\item [Categoría:] Asignaciones.
		\item [Descripción:] Se debe facilitar un mecanismo para que dos símbolos variables distintos referencien al mismo valor.
		Para ello se ha de facilitar un mecanismo para obtener la referencia de un símbolo variable, de forma que esta pueda ser usada en una asignación.
	\end {description}
%\end{framed}

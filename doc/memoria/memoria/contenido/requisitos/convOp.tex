\subsection{Operadores de conversión de tipo}
\subsubsection{Conversión a numérico}
%\begin{framed}
	\begin{description}
		\item [Número:] \cn
		\item [Nombre:] Conversión a numérico.
		\item [Categoría:] Operadores de conversión de tipo.
		\item [Descripción:] Se ha de facilitar un operador que permita convertir un dato a tipo aritmético. Esta conversión
		se deberá realizar en función al tipo de dato origen y de la forma descrita en el requisito en el que se hace referencia al mismo.
		El valor que tomará la operación deberá ser el dato tras la conversión de tipos. La conversión se podrá realizar a un número
      entro o flotante.
	\end {description}
%\end{framed}

\subsubsection{Conversión a lógico}
%\begin{framed}
	\begin{description}
		\item [Número:] \cn
		\item [Nombre:] Conversión a lógico.
		\item [Categoría:] Operadores de conversión de tipo.
		\item [Descripción:] Se ha de facilitar un operador que permita convertir un dato a tipo lógico. Esta conversión
		se deberá realizar en función al tipo de dato origen y de la forma descrita en el requisito en el que se hace referencia al mismo.
		El valor que tomará la operación deberá ser el dato tras la conversión de tipos.
	\end {description}
%\end{framed}

\subsubsection{Conversión a cadena de caracteres}
%\begin{framed}
	\begin{description}
		\item [Número:] \cn
		\item [Nombre:] Conversión a cadena de caracteres.
		\item [Categoría:] Operadores de conversión de tipo.
		\item [Descripción:] Se ha de facilitar un operador que permita convertir un dato a tipo cadena de caracteres. Esta conversión
		se deberá realizar en función al tipo de dato origen y de la forma descrita en el requisito en el que se hace referencia al mismo.
		El valor que tomará la operación deberá ser el dato tras la conversión de tipos.
	\end {description}
%\end{framed}

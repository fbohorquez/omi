\subsection{Procesos}

\subsubsection{Ejecutar comando}
%\begin{framed}
	\begin{description}
		\item [Número:] \cn
		\item [Nombre:] Ejecutar comando.
		\item [Categoría:] Procesos.
		\item [Descripción:] Se debe facilitar un operador que ejecute un comando dado. 
		Para ello se deberá usar el interprete de comandos definido por el entorno del sistema operativo. Este operador contemplará
		un único operando correspondiente al comando a ejecutar. Como valor se deberá tomar la cadena de caracteres correspondiente 
		a la salida del comando.
	\end{description}
%\end{framed}

\subsubsection{Evaluar cadena}
%\begin{framed}
	\begin{description}
		\item [Número:] \cn
		\item [Nombre:] Evaluar cadena.
		\item [Categoría:] Procesos.
		\item [Descripción:] Deberá existir un operador que utilice el interprete para procesar una cadena de caracteres escrita en 
		el léxico y con la sintaxis del propio lenguaje. Este operador tomará como valor una cadena de caracteres relativa a la salida 
		generada por la interpretación.
	\end{description}
%\end{framed}

\subsubsection{Bifurcación de proceso}
%\begin{framed}
	\begin{description}
		\item [Número:] \cn
		\item [Nombre:] Bifurcación de proceso.
		\item [Categoría:] Procesos.
		\item [Descripción:] Se necesita de un mecanismo que bifurque el flujo de ejecución mediante la creación de un proceso hijo.
		El intérprete deberá crear un proceso clonado de si mismo, cuya ejecución proseguirá en el mismo punto. El operador 
		de bifurcación tomará valor aritmético cero en el proceso hijo, mientras que en el padre tomará el valor aritmético
		correspondiente al identificador del proceso hijo.    
	\end{description}
%\end{framed}

\subsubsection{Espera entre procesos}
%\begin{framed}
	\begin{description}
		\item [Número:] \cn
		\item [Nombre:] Espera entre procesos.
		\item [Categoría:] Procesos.
		\item [Descripción:] Se necesita de un mecanismo que permita hacer que la ejecución de un proceso padre espere 
		a que todos o algunos de sus hijos finalicen su ejecución. Este podrá operar sobre un dato aritmético 
		que referenciará al identificador de proceso del hijo que se ha de esperar. El valor que tomará consistirá en
		el código correspondiente a la señal de salida producida por el último proceso finalizado.     
	\end{description}
%\end{framed}

\subsubsection{Obtener identificador de proceso}
%\begin{framed}
	\begin{description}
		\item [Número:] \cn
		\item [Nombre:] Obtener identificador de proceso.
		\item [Categoría:] Procesos.
		\item [Descripción:] Todo proceso tiene un identificador único en el sistema sobre el que se ejecuta. Se deberá
		disponer de un operador que tome el valor del identificador de proceso correspondiente al interprete.     
	\end{description}
%\end{framed}

\subsubsection{Obtener identificador de proceso padre}
%\begin{framed}
	\begin{description}
		\item [Número:] \cn
		\item [Nombre:] Obtener identificador de proceso padre.
		\item [Categoría:] Procesos.
		\item [Descripción:] Debe existir algún mecanismo que permita obtener el identificador del proceso padre cuando 
		el interprete se ejecuta como proceso hijo de otro.
	\end{description}
%\end{framed}

\subsubsection{Señales entre procesos}
%\begin{framed}
	\begin{description}
		\item [Número:] \cn
		\item [Nombre:] Señales entre procesos.
		\item [Categoría:] Procesos.
		\item [Descripción:] Debe existir algún mecanismo que permita mandar señales entre procesos. Estas señales se corresponderán con
      señales UNIX. Para mandar una señal a un proceso se deberá dar un identificador de proceso y un entero correspondiente a la
      señal a enviar.
	\end{description}
%\end{framed}

\subsubsection{Manejador de señales a procesos}
%\begin{framed}
	\begin{description}
		\item [Número:] \cn
		\item [Nombre:] Manejador de señales a procesos.
		\item [Categoría:] Procesos.
		\item [Descripción:] Debe existir algún mecanismo que permita especificar una función que será ejecutada cuando el proceso reciba una 
      determinada señal.
	\end{description}
%\end{framed}

\subsubsection{Llamar a función como proceso}
%\begin{framed}
	\begin{description}
		\item [Número:] \cn
		\item [Nombre:] Llamar a función como proceso.
		\item [Categoría:] Operadores sobre procesos.
		\item [Descripción:] Se precisa de operador que mediante una función y un listado de parámetros realice una llamada 
		a la misma mediante la creación de un proceso hijo.
	\end{description}
%\end{framed}

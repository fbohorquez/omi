\subsection{RunTree}

\subsubsection{Intérprete online}
%\begin{framed}
	\begin{description}
		\item [Número:] \cn
		\item [Nombre:] Intérprete online.
		\item [Categoría:] RunTree.
		\item [Descripción:] Se ha de disponer de una herramienta que permita
      hacer uso del intérprete desde el navegador. Esta herramienta tomará un código fuente 
      escrito en el lenguaje de programación y lo enviará a un intérprete OMI para su procesamiento. 
      Luego mostrará, paso a paso, información relativa al proceso de interpretación.
	\end {description}
%\end{framed}

\subsubsection{Escribir código fuente}
%\begin{framed}
	\begin{description}
		\item [Número:] \cn
		\item [Nombre:] Escribir código fuente.
		\item [Categoría:] RunTree.
		\item [Descripción:] La herramienta deberá de disponer de un campo de formulario en el que se pueda escribir el código fuente
      que será enviado para interpretar. Este campo deberá mostrar un resaltado de sintaxis.
	\end {description}
%\end{framed}

\subsubsection{Árbol sintáctico}
%\begin{framed}
	\begin{description}
		\item [Número:] \cn
		\item [Nombre:] Árbol sintáctico.
		\item [Categoría:] RunTree.
		\item [Descripción:] La herramienta deberá de mostrar el árbol sintáctico resultado de la interpretación y permitir la navegabilidad por este.
	\end {description}
%\end{framed}

\subsubsection{Tabla de símbolos}
%\begin{framed}
	\begin{description}
		\item [Número:] \cn
		\item [Nombre:] Tabla de símbolos.
		\item [Categoría:] RunTree.
		\item [Descripción:] La herramienta deberá de mostrar el estado de las tablas de símbolos en cada paso del proceso.
	\end {description}
%\end{framed}

\subsubsection{Salida de datos}
%\begin{framed}
	\begin{description}
		\item [Número:] \cn
		\item [Nombre:] Salida.
		\item [Categoría:] RunTree.
		\item [Descripción:] La herramienta deberá mostrar la salida generada por cada paso del proceso.
	\end {description}
%\end{framed}

\subsubsection{Entrada de datos}
%\begin{framed}
	\begin{description}
		\item [Número:] \cn
		\item [Nombre:] Entrada de datos.
		\item [Categoría:] RunTree.
		\item [Descripción:] La herramienta deberá solicitar al usuario la entrada de datos cunado el proceso de interpretación lo requiera.
	\end {description}
%\end{framed}

\subsubsection{Limpieza de salida}
%\begin{framed}
	\begin{description}
		\item [Número:] \cn
		\item [Nombre:] Limpieza de salida.
		\item [Categoría:] RunTree.
		\item [Descripción:] La herramienta deberá dar la posibilidad de limpiar la salida generada por el proceso de interpretación.
	\end {description}
%\end{framed}

\subsubsection{Consola de información}
%\begin{framed}
	\begin{description}
		\item [Número:] \cn
		\item [Nombre:] Consola de información.
		\item [Categoría:] RunTree.
		\item [Descripción:] La herramienta deberá de disponer de una sección en la que se muestre textualmente la operación llevada a cabo.
	\end {description}
%\end{framed}

\subsubsection{Ejecución paso a paso}
%\begin{framed}
	\begin{description}
		\item [Número:] \cn
		\item [Nombre:] Paso a paso.
		\item [Categoría:] RunTree.
		\item [Descripción:] La herramienta deberá de disponer de acciones que permitan avanzar la ejecución un paso.
	\end {description}
%\end{framed}

\subsubsection{Ejecución sentencia a sentencia}
%\begin{framed}
	\begin{description}
		\item [Número:] \cn
		\item [Nombre:] Sentencia a sentencia.
		\item [Categoría:] RunTree.
		\item [Descripción:] La herramienta deberá de disponer de acciones que permitan avanzar la ejecución una sentencia.
	\end {description}
%\end{framed}

\subsubsection{Ejecución automática}
%\begin{framed}
	\begin{description}
		\item [Número:] \cn
		\item [Nombre:] Ejecución automática.
		\item [Categoría:] RunTree.
		\item [Descripción:] La herramienta deberá de disponer de la capacidad de ejecutar paso a paso el programa pero de una forma automática.
	\end {description}
%\end{framed}

\subsubsection{Guardar código fuente}
%\begin{framed}
	\begin{description}
		\item [Número:] \cn
		\item [Nombre:] Guardar código fuente.
		\item [Categoría:] RunTree.
		\item [Descripción:] La herramienta deberá de disponer de la capacidad de guardar el código fuente escrito en un fichero local.
	\end {description}
%\end{framed}

\subsubsection{Abrir código fuente}
%\begin{framed}
	\begin{description}
		\item [Número:] \cn
		\item [Nombre:] Abrir código fuente.
		\item [Categoría:] RunTree.
		\item [Descripción:] La herramienta deberá de disponer de la capacidad de leer el código fuente escrito en un fichero local para enviarlo a interpretar.
	\end {description}
%\end{framed}

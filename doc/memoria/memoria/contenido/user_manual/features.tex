OMI representa una plataforma constituida por una serie de herramientas. Estas ayudan al aprendizaje en la aplicación práctica de la teoría
de autómatas y lenguajes formales para el desarrollo de un lenguaje de programación moderno. Sus características son las siguientes:

\begin{itemize}
   \item Intérprete OMI
   \begin {itemize}
      \item Abierto
      \item Modular
      \item Interactivo
      \item Configurable
      \item Paso de argumentos
      \item Ejecución en modo servidor
      \item Autodescriptivo.
   \end{itemize}
   \item Lenguaje OMI
   \begin {itemize}
      \item Propósito general
      \item Interpretado
      \item Sintaxis simple y cercana a los lenguajes modernos
      \item Tipado dinámico y débil
      \item Tipos de datos simples y compuestos
      \item Referencia de datos
      \item Funciones y operadores sobre los distintos tipos de datos
      \item Sentencias de control 
      \item Variable de ámbito global y local
      \item Definición de funciones
      \item Paso de parámetros por valor y por referencia
      \item Funciones de orden superior
      \item Clausura de funciones
      \item Funciones anónimas
      \item Aplicación parcial de funciones
      \item Decoradores
      \item Definición de clases de objetos
      \item Creación e instanciación de objetos
      \item Tipos de datos como clases de objetos
      \item Visibilidad de métodos y atributos
      \item Definición estática de métodos y atributos
      \item Polimorfismo
      \item Duck typing
      \item Herencia simple
      \item Métodos mágicos
      \item Dynamic binding
      \item Excepciones
      \item Evaluación por cortocircuito devolviendo último valor
      \item Operadores condicionales 
      \item Funciones de fechas y tiempo \\
      \item Funciones de creación y acceso a ficheros
      \item Concurrente
      \item Recolector de basura
   \end{itemize}
   \item Web OMI project.
   \begin{itemize}
      \item Contenido estático
      \item Acceso a la documentación del proyecto
      \item Navegación por las distintas clases y módulos que conforman el software
      \item Acceso a la descarga del software
      \item Información sobre el proyecto y la autoría
      \item Información de contacto
   \end{itemize}
   \item runTree
   \begin{itemize}
      \item Cliente OMI
      \item Alta disponibilidad y facilidad de acceso
      \item Interactivo
      \item Interfaz gráfica
      \item Descripción de diagramas
      \item Descripción del árbol sintáctico producido
      \item Descripción de las tablas de símbolos
      \item Descripción de la entrada/salida producidos
      \item Ayuda en pantalla
   \end{itemize}
\end{itemize}


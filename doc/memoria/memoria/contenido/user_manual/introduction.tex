OMI, acrónimo de Open Modular Interpreter, es un lenguaje para la creación de `scripts' que presenta un propósito general y es de código abierto.
El objetivo del proyecto OMI es servir de guía en la aplicación práctica de la teoría de compiladores e intérpretes. Para ello presenta un caso práctico 
en el que se construye un lenguaje de programación moderno a partir de los conceptos teóricos de autómatas y lenguajes formales. 

La plataforma OMI representa un sistema software desarrollado para la comunidad académica. Pretende ser una herramienta utilizada en el aprendizaje de
los sistemas intérpretes y traductores modernos, además de cualquier otro basado en los conceptos sobre los que estos se construyen. 

La plataforma OMI se constituye mediante los siguientes elementos:

\begin{itemize}
   \item Documentación completa del proceso aplicado en el desarrollo de un lenguaje de programación.
   \item Intérprete de un lenguaje de programación denominado OMI, el cual da nombre al proyecto.
   \item Aplicación web que permite consultar la documentación e interactuar con el intérprete. 
\end{itemize}

Este manual describe al lenguaje, las funciones que integra, y las características del intérprete. Además detalla cómo se 
estructura el contenido dentro de la aplicación web y las herramientas que la componen.



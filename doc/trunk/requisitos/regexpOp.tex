\subsection{Operadores sobre expresiones regulares}

\subsubsection{Crear expresión regular}
\begin{framed}
	\begin{description}
		\item [Número:] \cn
		\item [Nombre:] Operador creador de expresión regular.
		\item [Categoría:] Operadores sobre expresiones regulares.
		\item [Descripción:] Dada una cadena de caracteres se necesita de un operador que convierta esta en una expresión regular. El
		valor del operador será la expresión regular.
	\end {description}
\end{framed}

\subsubsection{Comprobar expresión regular}
\begin{framed}
	\begin{description}
		\item [Número:] \cn
		\item [Nombre:] Función comprobación de expresión regular.
		\item [Categoría:] Operadores sobre expresiones regulares.
		\item [Descripción:] Se precisa un operador que, dada una expresión regular y una cadena de caracteres,
		compruebe si esta pertenece al lenguaje definido por la expresión regular. Se
		deberá tomar como valor un dato de tipo lógico resultado de la operación.
		
		Para que el resultado sea positivo la cadena debe pertenecer al conjunto de palabras delimitadas
		por la expresión regular. Si tan solo existe correspondencia parcial el resultado será negativo.
	\end {description}
\end{framed}

\subsubsection{Buscar mediante expresión regular}
\begin{framed}
	\begin{description}
		\item [Número:] \cn
		\item [Nombre:] Función de búsqueda estructurada.
		\item [Categoría:] Operadores sobre expresiones regulares.
		\item [Descripción:] Se precisa de un mecanismo que lleve a cabo una búsqueda estructurada, es decir,
		obtener una estructura de datos array condicionada por una expresión regular denominada
		``patrón de búsqueda'' y una cadena de caracteres ``texto'' sobre la que se comprueba.

		En la ejecución de este operador se deberá
		buscar en ``texto'' subcadenas que pertenezcan al conjunto definido por la expresión regular, originando
		un array con cada una de las coincidencias, que será el valor que tome la operación.
		
		Adicionalmente la expresión regular podría estar formada por subexpresiones delimitadas
		por ``()''. En dicho caso se buscará en ``texto'' subcadenas que pertenezcan al
		conjunto delimitado por la expresión regular. Por cada subcadena encontrada se creará un array
		con las correspondencias de cada subexpresión. Cada uno de los arrays resultantes se
		deberán guardar en otro que será el valor que tome la operación.
		
		Si se utiliza una expresión regular formada por subexpresiones los arrays correspondientes
		a cada subcadena deberán tener índices numéricos. Sin embargo debe darse la posibilidad de
		especificar una lista ordenada de cadenas claves para crear un array asociativo.
 
		Además se ha de contemplar la búsqueda estructurada sobre un array de cadenas ``texto''.
	\end {description}
\end{framed}

\subsection{Operadores sobre cadenas}

\subsubsection{Concatenación}
\begin{framed}
	\begin{description}
		\item [Número:] \cn
		\item [Nombre:] Operador concatenación.
		\item [Categoría:] Operadores sobre cadena de caracteres.
		\item [Descripción:] La expresión que simboliza una operación de concatenación precisa de dos operandos que serán
		tratados como cadena de caracteres. El valor que tomará la expresión será la cadena resultante de concatenar ambas.
	\end {description}
\end{framed}

\subsubsection{explode}
\begin{framed}
	\begin{description}
		\item [Número:] \cn
		\item [Nombre:] Función explode.
		\item [Categoría:] Operadores sobre cadena de caracteres.
		\item [Descripción:] La función explode deberá tomar dos cadenas de caracteres como operandos denominadas ``texto'' y ``separador''.
		El valor será el array resultante de separar la cadena ``texto'' en diferentes cadenas en función la cadena ``separador''.
	\end {description}
\end{framed}

\subsubsection{implode}
\begin{framed}
	\begin{description}
		\item [Número:] \cn
		\item [Nombre:] Función implode.
		\item [Categoría:] Operadores sobre cadena de caracteres.
		\item [Descripción:] Representa la operación inversa a explode. La función implode deberá tomar como argumentos un array denominado
		``listado'' y una cadena denominada ``separador''. El valor de la expresión será una cadena resultado de concatenar cada uno
		de los elementos de ``listado'' separados por la cadena ``separador''.
	\end {description}
\end{framed}

\subsubsection{sprintf}
\begin{framed}
	\begin{description}
		\item [Número:] \cn
		\item [Nombre:] Función de formato.
		\item [Categoría:] Operadores sobre cadena de caracteres.
		\item [Descripción:] Se hace necesario un mecanismo que permita generar cadenas formateadas. Este deberá consistir
		en una cadena de caracteres denominada ``formato'' y un listado de expresiones. La cadena formato contendrá una serie de
		directivas de formato. Estas directivas serán sustituidas por el valor correspondiente, según posición, de la
		lista de expresiones. Cuando se realiza cada sustitución el valor es formateado según la directiva.
		
		Las directivas de formato tienen el siguiente forma:
		$$\%[operador][precisi\acute{o}n][formato]$$
		
		Los posibles operadores serán los siguientes:
		\begin{description}
			\item[+:] Fuerza la impresión del símbolo + cuando se formatean números positivos.
			\item[\^\ :] Convierte el caracteres a mayúsculas cuando se formatean cadenas de texto.
			\item[\#:] Añade el carácter 0x cuando se formatean números hexadecimales y el carácter 0 cuando se formatean octales.
		\end{description}
	
		La precisión se refiere al número de decimales que se imprimirán en el caso de formatear
		números o el número de caracteres en el caso de formatear cadenas.			

		El carácter de formato indica que tipo de formato se le dará al valor:
		\begin{description}
			\item[i|d:] Número entero.
			\item[u:] Sin signo.
			\item[f:] Coma flotante.
			\item[\%:] Carácter \%.
			\item[e:] Notación científica.
			\item[o:] Octal.
			\item[x:] Hexadecimal.
			\item[s|c:] Cadena de texto.
		\end{description}
	\end {description}
\end{framed}

\subsubsection{Buscar subcadena}
\begin{framed}
	\begin{description}
		\item [Número:] \cn
		\item [Nombre:] Función de búsqueda de subcadena.
		\item [Categoría:] Operadores sobre cadena de caracteres.
		\item [Descripción:] Esta es una función básica en el tratamiento de cadenas. Opera sobre dos argumentos
		que serán tratados como cadenas de caracteres, uno denominado ``texto'' y otro ``subcadena''. La función
		toma como valor un dato aritmético relativo a la posición de la primera ocurrencia de ``subcadena'' dentro
		de ``texto''. Si no se encuentra ningún resultado se tomará el valor nulo. Adicionalmente se puede dar otro
		operando denominado ``offset'' que simbolice la posición dentro de ``texto'' a partir de la cual se comenzará a buscar.
	\end {description}
\end{framed}

\subsubsection{Buscar y remplazar}
\begin{framed}
	\begin{description}
		\item [Número:] \cn
		\item [Nombre:] Función de remplazo de subcadena.
		\item [Categoría:] Operadores sobre cadena de caracteres.
		\item [Descripción:] Se necesita de un mecanismo para buscar y remplazar subcadenas dentro de otra.
		Este debe buscar las ocurrencias de una subcadena ``búsqueda'' en una
		cadena ``texto'', sustituyéndolas por una cadena de ``remplazo''. Esta función debe admitir
		el número máximo de sustituciones que se llevarán a cabo. Tras la ejecución debe tomar como valor la cadena
		resultante de sustituir en la cadena principal las ocurrencias de la subcadenas por la cadena de remplazo.

		Adicionalmente la subcadena de ``búsqueda'' puede ser una expresión regular, en cuyo caso
		se buscará subcadenas que pertenezcan al conjunto de las palabras definido por la
		expresión regular.
		
		Si se utiliza una expresión regular como patrón de búsqueda deberá ser posible
		utilizar en la cadena de remplazo parte de la subcadena
		que concuerda con la expresión regular. Para ello se ha de formar la expresión regular mediante
		subexpresiones delimitadas por ``()''. En la cadena de remplazo se debe poder hacer referencia,
		de forma posicional, a las subcadenas correspondientes a cada una de las subexpresiones.
	\end {description}
\end{framed}

\subsubsection{Remplazar subcadena}
\begin{framed}
	\begin{description}
		\item [Número:] \cn
		\item [Nombre:] Función de remplazo de subcadena mediante posiciones.
		\item [Categoría:] Operadores sobre cadena de caracteres.
		\item [Descripción:] Se necesita de un mecanismo para buscar y remplazar subcadenas dentro de otra.
		Este debe sustituir en una cadena ``texto'' la subcadena comprendida entre dos posiciones dadas por 
      expresiones numéricas, sustituyendo las subcadena correspondiente por una cadena de ``remplazo''. 
      Tras la ejecución debe tomar como valor la cadena
		resultante de sustituir, en la cadena principal, la subcadena correspondiente a las pociones dadas por la cadena de remplazo.
      
	\end {description}
\end{framed}

\subsubsection{Convertir a mayúsculas}
\begin{framed}
	\begin{description}
		\item [Número:] \cn
		\item [Nombre:] Función conversión a mayúsculas.
		\item [Categoría:] Operadores sobre cadena de caracteres.
		\item [Descripción:] Dada una cadena de caracteres se necesita de un mecanismo que convierta todos los caracteres
		alfabéticos que la conforman en mayúsculas. El valor que se tomará será la cadena resultante de la operación.
	\end {description}
\end{framed}

\subsubsection{Convertir a minúsculas}
\begin{framed}
	\begin{description}
		\item [Número:] \cn
		\item [Nombre:] Función conversión a minúsculas.
		\item [Categoría:] Operadores sobre cadena de caracteres.
		\item [Descripción:] Dada una cadena de caracteres se necesita de un mecanismo que convierta todos los caracteres
		alfabéticos que la conforman en minúsculas. El valor que se tomará será la cadena resultante de la operación.
	\end {description}
\end{framed}

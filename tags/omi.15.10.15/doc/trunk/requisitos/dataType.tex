\subsection{Tipos de datos}

\begin{framed}
	\begin{description}
		\item [Número:] \cn
		\item [Nombre:] Tipos de datos.
		\item [Categoría:] Tipos de datos.
		\item [Descripción:] El sistema debe ser capaz de interpretar y operar sobre diferentes tipos de datos. Las expresiones
		pueden tener un tipo de dato asociado que puede o no ser definido y fijo. Los tipos de datos pueden ser simples o compuestos.
		
		Un dato debe tratarse como diferente tipo en función el contexto en el que se utilice. Así un dato de un tipo concreto puede ser tratado como
		otro tipo de dato si fuese necesario. Un dato por si mismo siempre será considerado del tipo de dato con el que se creó, sin embargo
		cuando interviene en una operación es posible que se precise una conversión o equivalencia de tipos. Para ello debe tomar su valor como si de otro
		tipo se tratase. Si en la operación no es posible convertir el tipo en el tipo requerido se debe producir un error de tipos.
      
		Se debe establecer un mecanismo de conversión de tipos. La relación de conversión de tipo debe ser transitiva, así
		si un dato de tipo lógico puede verse como un dato de tipo aritmético y un dato aritmético como un cadena de caracteres, entonces el
		dato de tipo lógico también puede verse como una cadena.
   \end {description}
\end{framed}

\subsubsection{Nulo}
\begin{framed}
	\begin{description}
		\item [Número:] \cn
		\item [Nombre:] Tipo de dato nulo.
		\item [Categoría:] Tipo de dato.
		\item [Descripción:] Se debe contemplar el tipo de dato nulo. Este tipo de dato tendrá un único valor 
      posible. El valor nulo deberá representar un elemento no definido. Una expresión puede tomar el
      valor nulo cuando sea evaluada si se hace uso de elementos no definidos.
	\end {description}
\end{framed}

\subsubsection{Lógico}
\begin{framed}
	\begin{description}
		\item [Número:] \cn
		\item [Nombre:] Tipo de dato lógico.
		\item [Categoría:] Tipo de dato.
		\item [Descripción:] El sistema debe ser capaz de interpretar y operar sobre datos de tipo lógicos. Este tipo
		de dato sólo contempla dos posibles valores: falso y verdadero. Este será el tipo de dato más simple. Un dato lógico
		puede ser tratado como un tipo de dato aritmético tomándose falso como el valor cero, y verdadero como el valor uno.
		Los datos de tipo lógico deben ser elementos imprimibles.
	\end {description}
\end{framed}
\subsubsection{Aritmético}
\begin{framed}
	\begin{description}
		\item [Número:] \cn
		\item [Nombre:] Tipo de dato aritmético.
		\item [Categoría:] Tipo de dato.
		\item [Descripción:] El sistema debe ser capaz de interpretar y operar sobre datos de tipo aritméticos. Este tipo
		de dato contempla valores numéricos racionales. Todo dato aritmético además tiene asociado un valor lógico cuando se utiliza como este
		tipo de dato, tal que cualquier número distinto de cero tiene valor verdadero y el cero tiene el valor falso. Además cuando un dato aritmético
		es tratado como una cadena de caracteres se tomará la cadena que representa al número. Los datos de tipo aritmético deben
		ser elementos imprimibles.
	\end {description}
\end{framed}

\subsubsection{Cadena de caracteres}
\begin{framed}
	\begin{description}
		\item [Número:] \cn
		\item [Nombre:] Tipo de dato cadenas de caracteres
		\item [Categoría:] Tipo de dato.
		\item [Descripción:] El sistema debe ser capaz de interpretar y operar sobre datos de tipo cadena de caracteres. Este tipo
		de dato contempla cualquier sucesión de caracteres alfanuméricos, secuencias de escape, u otros signos o símbolos. Esta sucesión pude
		ser vacía. Una cadena de caracteres vendrá delimitada mediante comillas dobles o simples. Toda cadena de caracteres además tiene asociado
		un valor aritmético cuando se utiliza como este tipo de dato, tal que, si la cadena representa un número racional el valor será el del
		propio número, por otro lado, si la cadena
		no representa un número racional el valor aritmético de la misma será el número de caracteres que la conforman. Los datos de tipo cadena
		de caracteres deben ser elementos imprimibles. No se ha de considerar el tipo simple de dato carácter, pudiéndose tratar este como una cadena
		de un solo elemento.
	\end {description}
\end{framed}

\subsubsection{Expresiones regulares}
\begin{framed}
	\begin{description}
		\item [Número:] \cn
		\item [Nombre:] Tipo de dato expresión regular.
		\item [Categoría:] Tipo de dato.
		\item [Descripción:] El sistema debe ser capaz de interpretar y operar sobre datos de tipo expresión regular. Una expresión regular
		consiste en una cadena de caracteres que representan un patrón. Las expresiones regulares tendrán una sintaxis PERL. Una expresión regular
		se delimita mediante caracteres acento grave (\`\ ). El tipo de dato expresión regular no debe ser tratado como otro tipo de dato.
	\end {description}
\end{framed}

\subsubsection{Arrays}
\begin{framed}
	\begin{description}
		\item [Número:] \cn
		\item [Nombre:] Tipo de dato array.
		\item [Categoría:] Tipo de dato.
		\item [Descripción:] El sistema debe ser capaz de interpretar y operar sobre datos de tipo array. Este tipo
		de dato contempla cualquier sucesión de elementos. Estos elementos pueden ser pares de expresiones clave/valor donde la clave servirá
		para referenciar el valor dentro de la sucesión. También pueden ser simples expresiones por lo que se tomará automáticamente una clave
		numérica y secuencial según el orden del array y como valor el de la expresión. El significado semántico de las claves en un array puede
		ser numérico (array numérico) o cadenas de caracteres (array asociativo).  
		Los elementos del array serán ordenados por defecto de forma alfabética en función a la clave.
		Una definición de array se delimita mediante llaves y sus elementos se denotarán mediante un listado de expresiones o pares de estas.
		Los datos de tipo array deben ser elementos imprimibles. Un dato de tipo array solo puede ser tratado como un tipo de dato booleano,
		siendo falso si se encuentra vacío y verdadero en caso contrario.
	\end {description}
\end{framed}

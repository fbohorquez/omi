\subsection{Fechas y tiempo}

\subsubsection{Obtener fecha}
\begin{framed}
	\begin{description}
		\item [Número:] \cn
		\item [Nombre:] Obtener fecha.
		\item [Categoría:] Fechas y tiempo.
		\item [Descripción:] Es necesario disponer de un operador que dé formato a la fecha/hora local. Este operador 
		deberá tener como operando una expresión cadena de caracteres que contenga una serie de directivas de 
		formato. El valor que tomará la operación será una cadena de caracteres que represente la fecha/hora en
		el formato dado. Las directivas de formato serán las siguientes:
		\begin{description}
			\item[\%d:] Día del mes con dos dígitos.
			\item[\%j:] Día del mes sin ceros iniciales.
			\item[\%l:] Día de la semana de forma alfabética completa.
			\item[\%D:] Día de la semana de forma alfabética y con tres letras.
			\item[\%w:] Día de la semana de forma numérica (0-domingo,6-sábado).
			\item[\%z:] Día del año de forma numérica.
			\item[\%F:] Mes de forma alfabética.
			\item[\%m:] Mes de forma numérica con dos dígitos.
			\item[\%n:] Mes de forma numérica sin ceros iniciales.
			\item[\%M:] Mes de forma alfabética con tres letras.
			\item[\%Y:] Año con cuatro dígitos.
			\item[\%y:] Año con dos dígitos.
			\item[\%a:] Periodo del día (am/pm) en minúsculas.
			\item[\%A:] Periodo del día (am/pm) en mayúsculas.
			\item[\%g:] Hora en formato 12h sin ceros iniciales.
			\item[\%G:] Hora en formato 24h sin ceros iniciales.
			\item[\%h:] Hora en formato 12h con dos dígitos.
			\item[\%H:] Hora en formato 24h con dos dígitos.
			\item[\%i:] Minutos con dos dígitos.
			\item[\%U:] Segundos desde la Época Unix (1 de Enero del 1970 00:00:00 GMT).
			\item[\%\%:] Carácter \%.		
		\end{description}
	\end {description}
\end{framed}

\subsubsection{Tiempo Unix }
\begin{framed}
	\begin{description}
		\item [Número:] \cn
		\item [Nombre:] Operador time.
		\item [Categoría:] Fechas y tiempo.
		\item [Descripción:] Se precisa de un operador que calcule el número de segundos desde 
		la Época Unix (1 de Enero del 1970 00:00:00 GMT). Este operador no tendrá operandos y 
		tomará el valor aritmético correspondiente.		
	\end {description}
\end{framed}

\subsubsection{Parar ejecución}
\begin{framed}
	\begin{description}
		\item [Número:] \cn
		\item [Nombre:] Parar ejecución.
		\item [Categoría:] Fechas y tiempo.
		\item [Descripción:] Se deberá de proporcionar un mecanismo que permita suspender o bloquear
		la ejecución durante un tiempo dado. Constará de una expresión que
		represente el valor aritmético del tiempo en segundos.
	\end{description}
\end{framed}

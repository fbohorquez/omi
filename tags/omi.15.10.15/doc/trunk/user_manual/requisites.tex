El interprete OMI puede ser ejecutado sobre cualquier hardware actual. El software por si mismo
solo necesita unos pocos kilobytes de memoria para funcionar, sin embargo el hardware 
necesario dependerá en gran medida del código fuente que será interpretado.

Para usar el interprete OMI es necesario disponer de un sistema GNU/Linux. 
El intreprete depende de una serie de bibliotecas de programación. Dependiendo 
de la instalación que se realice será necesario la instalación previa de estas o no.

\begin{itemize}
\item readline
\item boost-regex
\end{itemize}

Por otro lado para el correcto uso de la aplicación web se precisa de un navegador web con soporte 
HTML5 y JavaScript.
